\section{Future Work}

This work presents a cohesive path toward the construction of the Asure network; however, we also consider this work to be a starting point for future research on decentralized social security systems. In this section, we identify and populate two categories of future work. This includes work that has been completed and merely awaits description and publication and open questions for improving the current protocols.

\subsection{On-going Work}

The following topics represent ongoing work.

\begin{itemize}
\item Plasma MVP implementation.
\item Mobile Application (Android, iOS)
\item Decentralized social security system research.
\item Asure-in-Ethereum interface contracts and protocols.
\item A full implementable Asure protocol specification.
\end{itemize}

\subsection{Open Questions}
There are still some areas for improvement that can positively affect the performance of the network. They can be addressed later on after collecting enough statistic upon which can be decided the importance and the necessity of making changes:

\begin{itemize}
\item A better solution for mass enter and exit strategies.
\item A secure solution for the data unavailability issue.
\item A more practical application of SNARK/STARK.
\item A better strategies for faster implementations of social security systems and new economic models.
\item A better primitive for the Proof-of-Stake Prove function, which is publicly-variable and transparent.
\end{itemize}

Since social security is only a specialized form of insurance it is obvious to also support decentralized insurances on the platform and that it is a good match to expand this platform for the market. The Asure ecosystem consists of the Asure network, the Asure protocol, the Asure platform is powered by potential third-party applications in the field of social security and the insurance environment. The acceptance of the ecosystem will grow steadily due to the resulting network effects and synergy effects. 