\section{Ausblick}

%\begin{itemize}
%\item Implementierung als Ethereum SmartContract und UI
%\item Skalierbare Loesung mittels Specialized Blockchain und Substrate und %Reduzierung Transaktionskosten
%\item Evaluation Stablecoins like DAI
%\item Freigeben von Nicht-abgerufenen Leistungen
%\item Reinitialisierung (Neustart des Systems)
%\item Haftung, Gavanance
%\end{itemize}
Das Hauptziel der vorliegenden Arbeit war es, eine Konzeption für die dezentralisierte Rente zu entwickeln. Abschließend kann angemerkt werden, dass es noch offene Punkte gibt die zur Diskussion anstehen.


\textbf{Ethereum Smart-Contract:} Als nächstes ist geplant die Spezification in diesem Paper in Smart-Contract auf Ethereum Netzwerk umzusetzen und der Community zum Review zur Verfügung zu stellen.\\
\textbf{Skalierbares Netzwerk:} Damit das System bei einer Unerstützung von mehreren Produkten und einer Vielzahl an Transaktionen umgehen kann, ist es notwendig ein skalierbares Netzwerk zu entwickeln speziell für dezentralisierte Rente und generell für dezentralisierte soziale Versicherungssysteme.\\
\textbf{Einsatz von Stablecoins:} Es fehlt noch eine Evaluation wie weit der Einsatz von Stablecoins interessant sein könnte um bei einem Umlage verfahren die Einlagen vor Volatilität zu schützen.\\
\textbf{Rentenauflösungen:} Es fehlt noch die Untersuchung was passiert wenn bestimmte Ansprüche Jahrelang nicht abgerufen werden.\\
\textbf{Reset:} Bei einer Vollständigen Auflösung wäre ein Reset des Systems interessant, dass die Initialisierungswerte einfach zurückgesetzt werden können.\\
\textbf{Haftung, Governance:} Die Frage nach der Haftung und Lenkung des Systems ist noch nicht 100\% geklärt.\\
\textbf{Asset Managent:} Neben reinem Umlage verfahren, können teil Einlagen getätigt werden, dies ist mit Risiko als auch mit Gewinnmöglichkeiten verbunden, für alle die an etwas mehr Risiko interessiert sind, wären auch solche Untersuchungen und Ergebnis von großem Interesse.\\
\textbf{Weitere Anwendungsfälle:} Neben einer dezentralisierten Rente, können auch andere Sozialversicherungen dezentral organisiert werden wie z.B. (Krankenversicherung, Unfallversicherung, Arbeitslosenversicherung u.s.w). Neben den bekannten Sozialversicherungen können auch die Zukünftigen wie Bedingungsloses Einkommen dezentral organisiert werden. Aus diesem Grund haben wir die Asure Stiftung in der Schweiz gegründet.

