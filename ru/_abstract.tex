\begin{abstract}
Социальное обеспечение является важным элементом экономического и политического развития общества. Однако в мире более 4.1 миллиарда человек не имеют доступа к системам социального обеспечения. \cite{ilo} И с другой стороны, существующие социальные системы имеют другие проблемы, которые необходимо преодолеть по демографическим причинам (например, коэффициент рождаемости 1.5 по сравнению со средним мировым показателем 2.5) или по причинам затрат (административные расходы более 50\% или даже больше, чем 100\%).
В настоящее время блокчейн Ethereum может выполнять максимум 1.3 миллиона транзакций в день. \cite{etherscan} Системы социального обеспечения частично основаны на нескольких сотнях миллионов транзакций в месяц и следовательно, они не могут быть устойчиво реализованы с использованием технологии блокчейна на сегодняшний день.

Системы социального обеспечения, основанные на блокчейне, имеют ряд преимуществ в сравнении с обычными системами социального обеспечения. Они обеспечивают постоянное и намного более высокое качество данных, используемых и хранящихся благодаря целостности процесса, неизменности и устойчивости системы, что позволяет проводить точный анализ в реальном времени. Прозрачность и неизменность транзакций обеспечивают защиту системы от манипуляций и коррупции. Используя блокчейн для устранения громоздкого и подверженного ошибкам ручного труда, можно достичь высокой степени автоматизации, экономичности, а также простоты отслеживания бизнес-процессов.

Прошлые разработки технологии блокчейн и их результаты показывают, что финансовые транзакции выполняемые через них, могут быть выполнены безопасно, автоматически и без каких либо посредников. Это говорит о том, что системы социального обеспечения, как системы обслуживающие население и использующие финансовые транзакции на основе правил, являются разумным вариантом использования публичного блокчейна. 

Блокчейн Ethereum соответствуюет таким решениям, как Casper и Sharding в конвейере, что в конечном итоге решит проблему масштабируемости на уровне 1. Даже в отношении людей, которые не имеют доступа к каким-либо системам социального обеспечения, количество транзакций необходимых для платежей и выплат, составляет по крайней мере количество вовлеченных людей, то есть миллиарды транзакций в месяц только лишь для пенсионной системы. 

Целью данного документа является изучение решения уровня 2 для оптимальной масштабируемости при сохранении всех преимуществ технологии блокчейн в отношении децентрализованных систем социального обеспечения. 
\newline\newline

\textbf{Примечание:} asure.network находится на стадии разработки. В настоящее время ведутся активные исследования и новые версии этого документа появятся на сайте http://asure.network
Для комментариев и предложений, свяжитесь с нами по адресу research@asure.network.

\end{abstract}


\newpage
% NEW SITE ---------------------------------------------------------------------------------