\section{Zusammenfassung}

Diese Arbeit verfolgt eine Vision wo die aufgeführten Probleme mit einem dezentralem Ansatz zu lösen als auch einen Weg aufzuzeigen wie ein Sozialversicherungssystem in der Zukunft funktionieren kann. 

Durch die freiwillige Teilnahme wurden Anreize entwickelt durch die das System als eine sehr interessante alternative anzusehen ist. Ein Anfang für eine transparente, faire und eine barrierefreie Altersvorsorge die Weltweit nutzbar ist.


%Problem - Lösungsansätze.
%der demografische wandel durch globalen und dezentralen ansatz lösen lässt.
%wo wir statt stable coin mechanismen uns nicht an den doller anhängen sondern an einen %referenzwert der eine kaufkraft wiederspiegelt.
%und können wir es so designen, dass menschen auch etwas mehr in der zukunft erhalten als %diese einbezahlt haben?


Halten sich alle an die gleichen regeln, wird lediglich, das System die Kaufkraft wieder geben. Bei einer 0\% Inflation und 0\% Deflation wird die selbe Anzahl an $Units$ die in das System eingezahlt wurden an die Nutzer ausgezahlt. Das bedeutet es gibt kein Gewinn und kein Verlust, sondern der Werterhalt wurde über Jahre einfach aufbewahrt. \footnote{ Es werden lediglich die Transaktionskosten anfallen, bei eigenem Netzwerk können diese auch noch weiter minimiert werden.}

Es wird davon ausgegangen, dass einige Nutzer sich für eine frühere Rente entscheiden werden, auch wenn es Teilverlust für einen Nutzer bedeutet. Einige Nutzer werden vollständigen Zugang zum System verlieren, weil z.B. der private Schlüssel verloren geht, die Rentenbeiträge werden nicht vererbt und nicht abgeholt. Nach einiger Zeit werden die eingezahlten Werte freigegeben und dadurch werden die anderen Nutzer im System profitieren.



