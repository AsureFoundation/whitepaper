\section{Introduction}

Asure is a protocol token whose blockchain runs on Layer 2 on Ethereum and a proof called Proof-of-Stake, where blocks are created by stakers that ensure the operation of social security systems. The Asure protocol provides an application and scaling service over a network of independent providers that do not rely on a single coordinator where: (1) Pay service providers to offer applications, (2) Node operators earn tokens by offering computing power.

By providing a network that enables the development of social security systems on top of blockchain technologies it is possible to create a decentralized autonomous organisation which functions in a cost-efficient way and is controlled by the community.


\subsection{Social security systems}

Social security is a mass transaction process and the automation that the blockchain technology brings with it is suitable for it. In Germany alone there are five social security systems and the pension alone causes more than 80 million transactions per month. 


Social security systems absorb many life risks, prevents extreme hardship and thus creates a social balance. In this way, it ensures social peace in the country.
\newline
There are various models of social security and social security financing. Social security systems can cover pension, health care, unemployment insurance, nursing and accident insurance. And the financing can be realized by found, tax, pay-as-you-go (PAYG) or a mixture of these.
In social security, PAYG refers to an unfunded system in which current contributors to the system pay the expenses for the current recipients. In a pure PAYG system, no reserves are accumulated and all contributions are paid out in the same period. The opposite of a PAYG system is a funded system, in which contributions are accumulated and paid out later when eligibility requirements are met.
\newline
Research results show that social security plays a central role in economic and political development in developing countries. People who do not have access to social security systems are at risk of falling into poverty if they are struck by a stroke of fate such as illness, crop failure or disability. They may then have to liquidate savings, sell livestock and other means of production and send their children to work instead of the school in order to finance daily expenses. \cite{erd}
\newline
People who enjoy basic social security are more willing to invest in education and physical capital, which entail additional risks, but also the prospect of income improvements. Empirical studies suggest that the existence of social security systems, especially in the informal sector, strengthens the propensity to invest and thus promotes economic growth precisely where this best contributes to poverty reduction. \cite{hcms}

\subsection{Blockchain}

The disruptive potential of blockchain becomes increasingly apparent. Blockchain technologies enable decentralization of the network allowing individuals to join and transact directly with the network. Every node in the system has a copy of the blockchain, thus removing the single point of failure therein. After the invention of the blockchain, the world was given the tools necessary to build real decentralized autonomous organizations (DAO). In such system, multiple authorities control different components and no single authority is fully trusted by all others. \cite{cammarden}
\newline

An Ethereum node’s maximum theoretical transaction processing capacity is over 1,000 transactions per second. Unfortunately, this is not the actual throughput due to Ethereum’s “gas limit”, which is currently around 6.7 million gas on average for each block. \cite{gaslimit}

\begin{itemize}
\item Scalability
\item Low performance
\item Energy consumption (Proof of work)
\end{itemize}