\section{Introduction}

%\begin{itemize}
%\item Private vs Sozialeversicherung
%\item Verfügbarkeit in Ländern ohne Sozialversicherung / Globale Versicherung
%\item Inflation
%\item Instrumentalisierung Politk
%\item Korruption und Fehlinvestiotionen
%\end{itemize}
%\subsubsection*{Altervorsorge}

Die Entwicklung in den letzten 150 Jahren hat dazu geführt, dass sich die Altersvorsorge durch den Familienverband hin zu größeren Gruppen (Staat, Kollektiv der Versichertengemeinschaft) verlagert hat. Rentensysteme sind ein wesentlicher Bestandteil in der Wirtschaftlichen Entwicklung unserer Welt, trotzdem gibt es 4.1 Mrd. Menschen ohne einen Sozialversicherungszugang. 

In den Industrienationen sind die Rentensysteme auf mehreren Säulen aufgebaut. Gesetzliche-, Betriebliche- und Private-Vorsorge damit wird der Altersarmut vorgebeugt. Wir möchten in diesem Papier eine Weitere Säule vorstellen und zwar die "Decentralized Pension". 


\subsubsection*{Problem: Staaten unterhalten}

Die jeweils aktive und leistungsfähige Generation betreibt private Altersvorsorge und beteiligt sich zusätzlich 

Die Inflation führt bei der Vermögenbildung im Rahmen der Altersvorsorge zu einem erheblichen Wertverlust. 
Rentensystem auf Basis des Kapitaldeckungsverfahren investieren Beiträge um eine Rendite zu erzielen, welche den Wertverlust durch die Inflation ausgleicht und im besten Fall zu einer Wertsteigerung führt.

Staatlich organisierte Rentensysteme können, im Gegensatz zu privatwirtschaftlich organisierten Rentensystemen, sowohl das Kapitaldeckungsverfahren als auch das Umlageverfahren verwenden. 

\subsubsection*{Problem: Kapitalgedeckte Rentensysteme}

Kapitalgedeckte Rentensysteme investieren die Beiträge und der Finanziert werden Rentensysteme durch Kapitaldeckungsverfahren oder Umlageverfahren.


\subsubsection*{Problem: Instrumentalisierung durch die Politik}

Probleme in Sozialversicherungen sind die Instrumentalisierung durch die Politik.

Die Rentensysteme werden in Wahlprogrammen gerne für die Wahlversprechen genommen um diese Anzupassen und somit auch die Wahlen zu Gewinnen, dies kann auch Nachteile haben, da eine Politik-Periode kurzlebiger ist als ein Rentenanspruch.

\subsubsection*{Problem: Demografische Wandel}

Wir sind davon überzeugt, dass die Umlageverfahren eine gutes Verfahren ist und dass in der globalen Welt auch die Probleme des Demografischen Wandels besser gelöst werden als auf Nationaler ebene. 

\subsubsection*{Problem: Rest Risiko}

Alle Rentensysteme sind nicht risikofrei, dies liegt in der Natur von Risiko orientierten Systemen, die Lösungsansätze liegen in der Risiko minimierung, durch verschiedene Ansätze wie Risikostreuung, Risikoübernahme durch das Land u.s.w.


\subsection{Our Approach: Decentralized Pension}

Die erste Frage die wir uns gestellt haben ist, kann eine Sozialversicherung ohne Einfluss von Firmen oder Regierungsbehörden funktionieren? Die wir nach einiger Zeit mit Ja beantwortet haben.

Als Inspiration haben wir uns an dem deutschen Umlageverfahren orientiert. Nach dem wir das deutsche Rentensystem auf Ethereum in Grundzügen implementiert haben. Haben wir die Herausforderungen gesehen die es zu Lösen galt.
Neben der Herausforderungen haben wir auch Chancen gesehen das System auch zu verbessern. z.B. bei der Vererbung des Rentenanspruchs.

Wir haben uns mit Experten aus dem Versicheurngs- und Rentensystemumfeld unterhalten um weitere Optimierungen am Model vorzunehmen.

-. Um GDPR Complince richtlinie werden keine externen und persönlichen Informationen benötigt.
-. Wir Verzichten auf die Altersangabe und
-. Rentenpunkte werden in der Basis auf 2.0 limitiert, dies hat den Hintergrund, dass keiner bei der Späteren Umverteilung einen unkalkulierbaren Anspruch haben darf.
-. Die Rente ist in Voller Höhe weiter Vererbbar


Für das Umlageverfahren sind verschiedene Voraussetzungen relevant. 

\begin{itemize}
\item Betragspflicht - sogt dafür, dass sicher gestellt wird, dass die Beiträge ralative Konstanz haben
\item Beitragswert der meistens Lohnorientiert ist - dies ermöglicht eine gewisse sicherheit und representiert in gewisser weise die Kaufkraft
\end{itemize}

Da wir aus geopolitischen gründen nicht möglich auf diese Voraussetzungen wert zu legen, deswegen haben wir alternativen erarbeitet.
Wir schaffen Anreize um bei der Beitragspflicht nachhelfen zu können und bei dem Beitragswert führen wir einen von Mitgliedern repräsentativen Referenzbeitragssatz der von den Mitgliedern dynamisch sich an das Verhalten anpasst.


Dezentrale Rente mittels Umlageverfahren und Incentivierung als SmartContract.
Global Verfügbar. Keine Verwaltungskosten (außer Tx Kosten) oder Provision

\subsection{Umlageverfahren}

\begin{itemize}
\item Beitragszahlung und Berechnung DPT
\item Rentenzahlung und Umverteilung der monatichen Beiträge
\item Rentenzahlung und Umverteilung der Rücklagen
\end{itemize}

\subsection{Incentivierung}

\begin{itemize}
\item Beitragsjahre = Bezugsjahre
\item Early Adopter Bonus
\item Bonus für letzte Teilnehmer im System
\end{itemize}


\subsection{Zielgruppe}

Die Zielgruppe für ein dezentralisiertes Rentensystem sind Menschen

\begin{itemize}
\item ohne Rentenzugang
\item wo es einen Rentenzugang gibt, dieser aber 
 \begin{itemize}
 \item Korrupt ist
 \item Intransparent ist
 \item zu hohe Inflation im Land existiert
 \item zu hohe Kosten verursacht
 \item keine guten Investitionen getätigt werden
 \item kein Vertrauen existiert
 \end{itemize}
\item die als weitere Ergänzung 
 \begin{itemize}
 \item Ihr Risiko über mehrere Risikoklassen streuen
 \item ein Crypto / Dezentralisiertes Produkt nutzen wollen
 \item die als digital Nomade leben
 \item nach alternativen suchen
 \end{itemize}
\end{itemize}

\subsection{Vorteile}

Die Vorteile eines dezentralisierten Rentensystems sind:

\begin{table}[htb]
\scalebox{0.7}{
\begin{tabular}{p{5cm} p{14cm}}
Decentralized & Mit hilfe der Blockchain technologie ist das System dezantralisiert verfügbar.  \\ \hline
Autonomous & Es sind keine Mitarbeiter Notwendig um das system betreiben zu müssen   \\ \hline
Permissionless & Der Zugang zu diesem System ist für jeden gleich zugänglich. \\ \hline
Transparent & Es ist Open-Source und jeder kann die Transaktionen einsehen und das System auf seine Verarbeitung prüfen. \\ \hline
Reduce costs & Durch die Automatisierung und Dezentralisierten Betrieb gibt es keinen weiteren Kostenaufwand ausser den Transaktionsgebühren. \\ \hline
Without any intermediary & Dieses System soll der Community zur Verfügung gestellt werden um Aufzuzeigen, dass es auch anders funktionieren kann und die Armut auf der Welt kann dadurch reduziert werden. \\ \hline
Corruption-proof & Es gibt keine Möglichkeit das Geld entwenden zu können. \\ \hline
Tamper-proof & Die erlaubten veränderungen des Systems sind den Mitgliedern überlassen. \\ \hline
Froud-proof & Betrug wird dadurch vermieden, dass wir keine Externen Informationen für den betrieb benötigen. \\ \hline
Haritable & Der Anspruch an eine Rente kann durch die Weitergabe des Privaten-Schlüssels übergeben werden.
\end{tabular}
}
\end{table}