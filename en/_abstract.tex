\begin{abstract}
Social security is an essential element in the economic and political development of societies. However, there are over 4.1 billion people worldwide without access to social security systems.\cite{ilo} And on the other hand, the existing social systems have other challenges that have to be overcome for demographic reasons (e.g. birth rates 1.5 compared to the world average of 2.5) or cost reasons (administrative costs of more than 50\% or even more than 100\%).
The Ethereum blockchain is currently only able to carry out a maximum of 1.3 million transactions per day.\cite{etherscan} Social security systems are based in part on several hundred million transactions per month and thus cannot be sustainably implemented using the blockchain as of today.

Blockchain-based social security systems have several advantages in comparison to conventional social security systems. They ensure a constant and much higher quality of the data used and stored through process integrity, immutability and the sustainability of the system, enabling accurate real-time analysis of those. The transparency and immutability of the transactions ensure the system's security against manipulation and corruption. By using Blockchain to remove the cumbersome and error-prone manual labor it is possible to achieve a high degree of automation, cost-efficiency, as well as easy to follow business processes.

The past developments of blockchain technology and their results show that financial transactions executed through them can be carried out securely, automatically and without intermediaries. This suggests that social security systems, as systems serving the public and using rule-based financial transactions, are a reasonable use-case for public blockchains. 

The Ethereum Blockchain corresponding solutions such as Casper, and Sharding in the pipeline that will eventually solve the scalability problem on Layer 1. Even regarding the people that don't have access to any social security systems the number of transactions required for pay-ins and payouts amounts to at least the number of people involved, i.e billions of transactions on a monthly basis for the pension system alone. 

The aim of this paper is to examine a Layer-2 solution for optimal scalability while maintaining all the benefits of blockchain technology regarding decentralized social security systems. 
\newline\newline

\textbf{Note:} asure.network is a work in progress. Active research is under way, and new versions of this paper will appear at http://asure.network
For comments and suggestions, contact us at research@asure.network.

\end{abstract}


\newpage
% NEW SITE ---------------------------------------------------------------------------------