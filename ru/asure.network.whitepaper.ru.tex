\documentclass[12pt,a4paper]{article}
\usepackage[utf8]{inputenc}
\usepackage[russian]{babel}
\usepackage{amsmath}
\usepackage{amsfonts}
\usepackage{amssymb}
\usepackage{graphicx}
\usepackage{tabularx}
\usepackage{url}
\usepackage{float}
\usepackage{color} 
\usepackage[sort=use]{glossaries-extra}

%\definecolor{pagecolor}{rgb}{0.8,0.9,0.9}

\usepackage[tocflat]{tocstyle}
\usetocstyle{standard}

%MVP, EWASM, TPS, ABCI, SDK, WebAssembly, DAO, DEPOT, ZK, PoS, Plasma

\newglossaryentry{evm}{name={EVM},plural={EVMs},
description={Виртуальная машина Ethereum предназначена для использования в качестве среды выполнения для смарт контрактов на основе Ethereum.}}
\newglossaryentry{blockchain}{name={Blockchain},plural={Blockchains},
description={Система, в которой ведется учет транзакций на нескольких компьютерах, которые связаны между собой в одноранговой сети.}}
\newglossaryentry{ethereum}{name={Ethereum},plural={Ethereum},
description={Децентрализованная программная платформа, которая включает в себя смарт контакрты и распределенные приложения (ĐApps).}}
\newglossaryentry{eth}{name={ETH},plural={eth},
description={Родной токен блокчейна Ethereum.}}
\newglossaryentry{btc}{name={BTC},plural={BTC},
description={Родной токен блокчейна Bitcoin.}}
\newglossaryentry{erc20}{name={ERC20},plural={ERC20},
description={Технический стандарт, используемый для смарт контрактов в блокчейне Ethereum для реализации токенов.}}
\newglossaryentry{smartcontract}{name={SmartContract},plural={SmartContract},
description={Смарт контракт это компьютерный протокол, предназначенный для упрощения, проверки или обеспечения выполнения согласования или исполнения контракта в цифровом виде. Смарт контракты позволяют совершать заслуживающие доверия транзакции без участия третьих лиц. Эти транзакции отслеживаются и необратимы.}}
\newglossaryentry{account}{name={Account},plural={accounts},
description={Хеш открытого ключа, который может содержать значения. Доступ возможен только при наличии соответствующего закрытого ключа.}}
\newglossaryentry{gdpr}{name={GDPR},plural={GDPR},
description={Общее положение о защите данных (ЕС) 2016/679 ("GDPR") это положение в законодательстве ЕС о защите данных и конфиденциальности для всех лиц в Европейском союзе (ЕС).}}
\newglossaryentry{payg}{name={PAYG},plural={PAYG},
description={метод финансирования социального страхования, в особенности пенсионного обеспечения, а также медицинского страхования и страхования по безработице. Уплаченные взносы используются непосредственно для финансирования бенефициаров, то есть они выплачиваются обратно им.}}


\begin{document}

%\pagecolor{pagecolor}
\pagenumbering{gobble}% Remove page numbers (and reset to 1)
\clearpage

% NEW SITE ---------------------------------------------------------------------------------
\begin{figure}
    \centering
    \includegraphics[width=2.0in]{img/logo.png}
\end{figure}

\title{ASURE: FIRST SCALABLE BLOCKCHAIN NETWORK FOR DECENTRALIZED SOCIAL SECURITY SYSTEMS}
\author{Paul Mizel, Fabian Raetz and Gamal Schmuck \\Asure Foundation}
\date{October 1, 2019}
\maketitle

\vskip 2.5in

\begin{quote}
	\centering
	\url{https://asure.network}
\end{quote}

\newpage 
% NEW SITE ---------------------------------------------------------------------------------

\pagenumbering{arabic}% Arabic page numbers (and reset to 1)
\newpage

\begin{abstract}

TODO

\end{abstract}


\section*{Keywords}
TODO; TODO; TODO



% list all entries
\printunsrtglossaries
\newpage
\tableofcontents

\newpage
% NEW SITE ---------------------------------------------------------------------------------
\section{Introduction}

\begin{itemize}
\item Private vs Sozialeversicherung
\item Verfügbarkeit in Ländern ohne Sozialversicherung / Globale Versicherung
\item Inflation
\item Instrumentalisierung Politk
\item Korruption und Fehlinvestiotionen
\end{itemize}

Die Entwicklung in den letzten 150 Jahren hat dazu geführt, dass sich die Altersvorsorge durch den Familienverband hin zu größeren Gruppen (Staat, Kollektiv der Versichertengemeinschaft) verlagert hat. 

Staaten unterhalten 

Die jeweils aktive und leistungsfähige Generation betreibt private Altersvorsorge und beteiligt sich zusätzlich 

Die Inflation führt bei der Vermögenbildung im Rahmen der Altersvorsorge zu einem erheblichen Wertverlust. 
Rentensystem auf Basis des Kapitaldeckungsverfahren investieren Beiträge um eine Rendite zu erzielen, welche den Wertverlust durch die Inflation ausgleicht und im besten Fall zu einer Wertsteigerung führt.





Staatlich organisierte Rentensysteme können, im Gegensatz zu privatwirtschaftlich organisierten Rentensystemen, sowohl das Kapitaldeckungsverfahren als auch das Umlageverfahren verwenden. 



Kapitalgedeckte Rentensysteme investieren die Beiträge und 

 der Finanziert werden Rentensysteme durch Kapitaldeckungsverfahren oder Umlageverfahren.



Altervorsorge 

Probleme in Sozialversicherungen sind die Instrumentalisierung durch die Politik.

Umlageverfahren sind gut 

\subsection{Our Approach: Decentralized Pension}

Die erste Frage die wir uns gestellt haben ist, kann eine Sozialversicherung ohne Einfluss von Firmen oder Regierungsbehörden funktionieren? Die wir nach einiger Zeit mit Ja beantwortet haben.

Als Inspiration haben wir uns an dem deutschen Umlageverfahren orientiert. Nach dem wir das deutsche Rentensystem auf Ethereum in Grundzügen implementiert haben. Haben wir die Herausforderungen gesehen die es zu Lösen galt.
Neben der Herausforderungen haben wir auch Chancen gesehen das System auch zu verbessern. z.B. bei der Vererbung des Rentenanspruchs.

Wir haben uns mit Experten aus dem Versicheurngs- und Rentensystemumfeld unterhalten um weitere Optimierungen am Model vorzunehmen.

-. Um GDPR Complince richtlinie werden keine externen und persönlichen Informationen benötigt.
-. Wir Verzichten auf die Altersangabe und
-. Rentenpunkte werden in der Basis auf 2.0 limitiert, dies hat den Hintergrund, dass keiner bei der Späteren Umverteilung einen unkalkulierbaren Anspruch haben darf.
-. Die Rente ist in Voller Höhe weiter Vererbbar


Für das Umlageverfahren sind verschiedene Voraussetzungen relevant. 

\begin{itemize}
\item Betragspflicht - sogt dafür, dass sicher gestellt wird, dass die Beiträge ralative Konstanz haben
\item Beitragswert der meistens Lohnorientiert ist - dies ermöglicht eine gewisse sicherheit und representiert in gewisser weise die Kaufkraft
\end{itemize}

Da wir aus geopolitischen gründen nicht möglich auf diese Voraussetzungen wert zu legen, deswegen haben wir alternativen erarbeitet.
Wir schaffen Anreize um bei der Beitragspflicht nachhelfen zu können und bei dem Beitragswert führen wir einen von Mitgliedern representativen Referenzbeitragssatz der von den Mitgliedern dynamisch sich an das Verhalten anpasst.

Die Vorteile dieses Systems sind:

\begin{table}[]
\scalebox{0.7}{
\begin{tabular}{p{5cm} p{14cm}}
Decentralized & Mit hilfe der Blockchain technologie ist das System dezantralisiert verfügbar.  \\ \hline
Autonomous & Es sind keine Mitarbeiter Notwendig um das system betreiben zu müssen   \\ \hline
Permissionless & Der Zugang zu diesem System ist für jeden gleich zugänglich. \\ \hline
Transparent & Es ist Open-Source und jeder kann die Transaktionen einsehen und das System auf seine Verarbeitung prüfen. \\ \hline
Reduce costs & Durch die Automatisierung und Dezentralisierten Betrieb gibt es keinen weiteren Kostenaufwand ausser den Transaktionsgebühren. \\ \hline
Without any intermediary & Dieses System soll der Community zur Verfügung gestellt werden um Aufzuzeigen, dass es auch anders funktionieren kann und die Armut auf der Welt kann dadurch reduziert werden. \\ \hline
Corruption-proof & Es gibt keine Möglichkeit das Geld entwenden zu können. \\ \hline
Tamper-proof & Die erlaubten veränderungen des Systems sind den Mitgliedern überlassen. \\ \hline
Froud-proof & Betrug wird dadurch vermieden, dass wir keine Externen Informationen für den betrieb benötigen. \\ \hline
Haritable & Der Anspruch an eine Rente kann durch die Weitergabe des Privaten-Schlüssels übergeben werden.
\end{tabular}
}
\end{table}




Dezentrale Rente mittels Umlageverfahren und Incentivierung als SmartContract.
Global Verfügbar. Keine Verwaltungskosten (außer Tx Kosten) oder Provision

\subsubsection{Umlageverfahren}

\begin{itemize}
\item Beitragszahlung und Berechnung DPT
\item Rentenzahlung und Umverteilung der monatichen Beiträge
\item Rentenzahlung und Umverteilung der Rücklagen
\end{itemize}

\subsubsection{Incentivierung}

\begin{itemize}
\item Beitragsjahre = Bezugsjahre
\item Early Adopter Bonus
\item Bonus für letzte Teilnehmer im System
\end{itemize}

%\section{Decentralized social security}
There are different requirements depending on the type of social insurance. This also has an impact on the plasma implementation. This chapter will take a closer look on the different requirements of these systems. 

\subsection{Pension}
A pension system consists of a number of contributors and pensioners. Every contributor pays a premium each month. In some systems, the premiums get paid by the company on behalf of the contributor which would mean a massive reduction of transactions needed.
On the other side, all pensioners get their pension from the pension system. This usually happens at a fixed date and all pensioners get paid at the same time. This makes it an ideal use case for mass payout transactions.

\subsection{Unemployment}
Unemployment insurance is the protection against loss of work. Participants who have a job pay a contribution where in case of loss of work the time is bridged by the contributors to find a job again.

\subsection{Health care}
The parties in healthcare are diverse - there are insured persons who make deposits, there are doctors, hospitals, pharmacies and other service providers who issue invoices, these can be offset against the system or via the insured person who submits the invoices to the system and gets the costs reimbursed. Here there are different possibilities how you can realize the processing in batches, the insured can submit the accumulated invoices at the end of the year or the doctors, hospitals, pharmacies and other service providers can also submit their collective invoices in batches.

\subsection{Other social security schemes}
The variable requirements show that the requirements for implementation are manifold and that there are a wide variety of variants for parties with different interests. As a result, it is a challenge to design the side-chains with mass deposits and mass exits.


%\newpage
\section{Asure Network}

The Asure protocol builds upon three components.

\begin{itemize}
\item \textbf{Scaled Blockchain Network:} We provide an abstraction for network of independent providers to offer application services. Later, we present the Asure protocol as an incentivized, auditable and verifiable construction.
\item \textbf{Platform:} By providing a platform that enables the development of social security systems on top of blockchain technologies.
\item \textbf{Protocol:} Social security systems exist in many different forms and Asure is providing general design specifications on how to implement these applications on blockchain.
halölo

\end{itemize}

\begin{figure}
    \centering
    \includegraphics[width=4.0in]{architecture.png}
    \caption{Asure architecture}
    \label{fig:asure_architecture}
\end{figure}

\subsection{Plasma}

Plasma is a framework. Plasma was very recently introduced and is among the more promising proposed solutions to scalable computation on the blockchain. Plasma can provide real scalability for ethereum applications. 
\newline
Plasma is an application-specific sidechain protocol.

Although the proposed Layer 2 protocols are still at an early stage and have not been described in detail, they essentially work by sending transactions to an off-chain platform. 

In order to raise the limits of layer 1 even further in order to effectively operate the social security system user group, layer-2 scaling is considered to be the most likely solution. Plasma is a good approach to the challenges of most.

\subsection{Sidechain}

Your chain will connect to Ethereum, inheriting its security and accountability. Your chain will be a sidechain.
\newline
This setup is the best of both worlds: From one part, it frees you from having to deal with securing your chain. For that, you can depend on the layer 1 chain.
\newline
From the other side, you can design your sidechain to match your needs.


\begin{figure}
    \centering
    \includegraphics[width=4.0in]{side-chain.png}
    \caption{Asure side chains}
    \label{fig:side_chain}
\end{figure}

\subsection{Protocol}

Plasma is not an implementation. It’s a protocol, or a set of guidelines. As long as you follow the specifications, your sidechain will be plasma.

\subsection{Plasma + EVM}

EVM provides such a powerful turing-complete computation so that ethereum can run a general program, also known as smart contract. Plasma EVM is a new version of Plasma that can execute EVM in plasma chain, and its clients can be based on current ethereum clients (go-ethereum, py-evm, parity). We propose state-enforceable Plasma construction to guarantee only valid state submitted to root chain, providing a way to enter and exit account storage between two chains because each chain has identical architecture. Another benefit is that ethereum development tools can be also used in plasma chain.

\subsection{Plasma + WASM, eWASM, *WASM}

WASM seems more secure, webassembly is backed by Google, Apple and Microsoft, the community is also active, it's gonna be a widely used platform. 
So embrace WASM will be a really good choice.

eWASM is a restricted subset of WASM (WebAssembly) to be used for contracts in Ethereum.

%\newpage
\section{Asure Blockchain}
From a technical point of view social security systems can be described as a number of rule-based (financial) transactions which are executed between a (usually) slightly changing total of different parties under the condition to maintain an equilibrium between deposited and withdrawn value over a period of time. Such a system can be implemented digitally by creating a blockchain-system, which supports smart contracts and cryptocurrencies.

Conventional social security systems currently generate up to hundreds million transactions per month, depending on the number of parties involved and the specific social security use-case. 

\begin{table}[H]
\centering
\begin{tabular}{lp{.2\textwidth}l}
  Monthly pension premiums & = 54.445 Mio\\
  Monthly pensions & = 25.646 Mio\\\hline
  Monthly pension transactions & = 80.091 Mio
\end{tabular}
\caption{\label{tab:table-name} For example the German statutory pension system: \cite{eckzahlen}}
\end{table}

In order to develop a blockchain system that can process these transactions, it is necessary to increase the achievable transaction throughput of the system and automatic batch processing within a transaction to reduce the number of total transactions to a minimum.

Both requirements can be addressed by the use of side-chains, as specified in the Plasma Framework. The Asure Blockchain functions as the scalable side-chain of the Asure Plasma implementation. It is the root-chain of the Asure Network and lays the foundation for optimal scalability regarding blockchain-based social security systems. 

Assets transferred from the Ethereum Blockchain to one of the Asure side-chains, are locked up in the Asure Plasma Contract on the Ethereum Blockchain until an exit transaction on the Ethereum Blockchain is executed. According to the Plasma MVP specifications, an equivalent of this value is created through the use of the operator design pattern (Proof-Of-Authority) on the Asure Blockchain and assigned to the user.

The available assets on the Asure Blockchain can then be used for transactions within the system. Consensus between all node providers within the Asure Blockchain is reached through a proof-of-stake consensus algorithm by using an adapted version of the Tendermint consensus engine. \cite{tendermint} Tendermint can handle transaction volume up to 10,000 transactions per second. With the help of zones and sharding concepts, this size can be increased by a factor of 1000. This would ensure the sustainable operation of social security on blockchain. \cite{tendermint_bench}

\begin{figure}[H]
    \centering
    \includegraphics[width=4.0in]{img/architecture.png}
    \caption{Asure architecture}
    \label{fig:asure_architecture}
\end{figure}

The Asure Blockchain has several fundamentals. 

\subsection{Security}
The Asure Blockchain includes several features that protect it against such attacks as unauthorized spending, double spending, forging assets, and tampering with the blockchain. 

Each block added to the blockchain, starting with the block containing a particular transaction, is referred to as a confirmation of that transaction. Ideally, recipients and senders receiving payments should wait until at least one confirmation has been distributed across the network before assuming that the payment has been made. The more confirmations the recipient waits, the more difficult it is for an attacker to successfully reverse the transaction in a blockchain unless the attacker controls more than half of the total network performance, in which case it is called a 51\% attack. This construction is not designed to prevent 51\% of attacks, but rather to encourage block propagation. 

\subsection{Consensus algorithm}
There are different versions for proof algorithms. Proof-of-work is highly criticized because of enormous power consumption.\cite{hackernoon} Long-term acceptance and community movement is moving towards proof-of-stake where validators create the blocks and are rewarded for doing the correct job. The Asure blockchain will use a Proof-of-Stake (PoS) consensus algorithm. It will use in the first MVP implementation the Tendermint consensus engine.\cite{tendermint}

\subsection{Privacy with (ZK-SNARKS and ZK-STARK)}
Among other things, the Asure Blockchain takes into account privacy aspects that have an enormous relevance in relation to social security.

ZK-SNARKS (Zero-Knowledge Succinct Non-interactive Argument of Knowledge) offers the possibility to carry out anonymous transactions. The ZK-SNARKS are not resistant to Quantum Computing. ZK-STARK (Zero-Knowledge Scalable Transparent Argument of Knowledge) is the latest innovation aimed to achieve privacy on the blockchain with the use of fast, scalable computations and is resistant to Quantum Computing. \cite{iacr}

Since Ethereum is also researching in Layer 1 in this area, it will be possible for social security transactions to remain anonymous for those insured. \cite{ethereum_zksnarks}

The state of Zero-Knowledge technologies is not yet entirely practicable, but this will change in the future.

\subsection{EVM, WASM, eWASM, *WASM}
EVM provides a turing-complete computation so that Ethereum can run a general program, also known as a smart contract. Plasma EVM is a new version of Plasma that can execute EVM in plasma chain, and its clients can be based on current Ethereum clients (go-ethereum, py-evm, parity). We propose state-enforceable Plasma construction to guarantee only valid state submitted to root-chain, providing a way to enter and exit account storage between two chains because each chain has identical architecture. Another benefit is that Ethereum development tools can also be used in plasma chain.

eWASM is just an Ethereum "flavored" subset of Web Assembly, which is binary instruction format. eWASM  relies on instructions that are very close to real-world CPU. The performance improvements are significant and seem more secure. WebAssembly is backed by Mozilla, Google, Apple, and Microsoft, the community is also active, it's gonna be a widely used web standard.  

The Ethereum Blockchain processes about 15 transactions per second (TPS), which is not sufficient for the implementation of a social security system. The improvements to Ethereum (also called Layer 1), which are currently in progress, should significantly increase the number of TPS. Among the improvements are a Proof-of-Stake (PoS) based consensus algorithm, sharding, and by the introduction of eWASM - a WebAssembly based virtual machine.

\subsection{Further technologies}
Parity Substrate is a high-level framework for creating cryptocurrencies and other decentralized systems using the latest research in blockchain technology. 

Cosmos-SDK is a blockchain framework to allow developers to easily create custom interoperable blockchain applications within the Cosmos Network without having to recreate common blockchain functionality, thus removing the complexity of building a Tendermint ABCI application. We envision the SDK as the npm-like framework to build secure blockchain applications on top of Tendermint.

LotionJS aims to make writing new blockchains fast. It builds on top of Tendermint using the ABCI protocol. Lotion lets you write secure, scalable applications that can easily interoperate with other blockchains on the Cosmos Network.



\section{Asure Platform}

The Asure platform consists of components that provide the network and protocol for the use and construction of social security systems, including the Client, SDKs, tools and frontend applications. The purpose of the platform is to create an ecosystem in which social security systems can be developed, tested, simulated, managed and productively used as quickly as possible.

\subsection{Client}
Main client is the entry point into the Asure network, capable of running a node. Nodes are connected to each other in a peer-to-peer network and relay new information by gossip protocol. Each node keeps a complete copy of a totally ordered sequence of events in the Asure blockchain. The nodes are used to form and operate the Asure network and ensure that the transactions are included in the Asure blockchain.

\subsection{Software Development Kits (SDKs) }
The SDK provides standardized features on which applications can be built. Our primary goal is to simplify the development of new ecosystem solutions so that they require little to no developer support.

\subsection{Tools}
The tools support the creation, testing, and simulation of created solutions on the Asure network and blockchain and speed up the development process.

\subsection{Frontend applications}
In order to achieve user acceptance, the blockchain standard applications are provided, such as blockchain-explorer, pool, mobile-apps (Android, iOS,) with a wallet to make the experience of mobile payments on a  global scale possible, as well as unlocking the full potential of mobile commerce.


%\newpage

%\section{Asure TGE}

The Asure TGE will happen in two rounds. The first round will take place in the beginning of 2019. The second round will happen later in 2019. Tokens will be ERC20 / ERC223 compatible and limited in supply by 100.000.000. In total, we will sell 45\% of all ASR utility tokens ("ASR") through the TGE. 

\begin{table}[H]
\begin{tabular}{lp{.6\textwidth}l}
  Token name & Asure Token \\  
  Ticker & ASR\\
  Token issuer & Asure Foundation\\
  Token type & ERC20 / ERC223 \\
  Token for sale & 45.000.000 ASR \\
  Total token supply & 100.000.000 ASR \\
  Accepted currencies & ETH \\
  Exchange Rate & 1 ASR = \$ 1.00 (ETH equivalent) \\
  Minimum Contribution & \$ 100 (ETH equivalent) \\\hline  
  
  Pre-Sale & 19\textsuperscript{th} Feb - 12\textsuperscript{th} Mar 2019 \\
  Pre-Sale Cap & \$ 5.000.000 | 10 Million ASR\\
  Pre-Sale Terms & First week 40\% bonus\newline
                   Second week 30\% bonus\newline
                   Third week 20\%  bonus\\\hline
  
  Main-Sale & 13\textsuperscript{th} Sep - 25\textsuperscript{th} Dec 2019 \\  
  Main-Sale Cap & \$ 35.000.000 | 35 Million ASR\\
  Main-Sale Terms & First month 15\% bonus\newline
                    Second month 5\% bonus\newline
                    Third month 0\% bonus\\\hline

  Special Bonus & 
From \$ 5 to \$ 25 thousand  3\%\newline
From \$ 25 to \$ 100 thousand  5\%\newline
From \$ 100 thousand 10\%\\\hline

  Listing & ASR tokens will be listed on crypto exchanges \\
  Token Holder Benefits & ASR token serves as the access to the Asure network by\newline
    \textbf{Network Validators:} Block rewards\newline
        \textbf{Service Providers:} Service rewards\\
  Token Trade Limitation & Only Team and Advisors have vesting and sales lock-in periods \\
  Hint & All unsold tokens in public TGE  will be burned \\\hline  
  
  Hardcap & \$ 40.000.000
  
\end{tabular}
\caption{\label{tab:table-name}Token details}
\end{table}







\subsection{Token}

The ASR token is basically a utility token. It is implemented as an Ethereum smart contract and supports the ERC-20 / ERC-223 token standards. The token will be used by Network Validators as well as Service Providers to participate as stakers in the proof-of-stake consensus mechanisms of the Asure network. It is an incentive to work correctly.

\begin{figure}[H]
    \centering
    \includegraphics[width=4.0in]{img/staking.png}
    \caption{What Is Benefit For Our Token Holders}
    \label{fig:asure_architecture}
\end{figure}

\textbf{Network validators:}
Network validators need to stake a to be delivered amount of ASR tokens to validate transactions within the Asure network. In return the network validators receive the networks transaction fees in the form of ASR as an incentive for validating correct blocks. In case of fraudulent behaviour of a validator, the fraudulent validator will lose its stake to the network.

\textbf{Service providers:}
The ASR token serves as an incentive for service providers within the Asure platform to work correctly and as advertised within their SLA’s. Service providers must deposit ASR tokens which are retained in the event of non-compliance with the corresponding SLA. A service within the Asure network could be e.g. oracles, products, reinsurance and sales activity.\newline

Demand for ASR tokens will be created by the two following mechanisms: With the growth of the Asure network, blockchain or platform more Network Validators and Service Providers will need to stake ASR tokens. The more Network Validators and Service Providers want to earn, the more ASR tokens must be staked.\newline
The ASR token will be listed on crypto exchanges for public trading so that service providers of the Asure Network, blockchain or platform can buy and sell ASR tokens.
\newline\newline



\subsection{Token allocation}

It is very important that the community understands how Asures funds are going to be invested in the future in order to contribute to the idea of creating a world with an open decentralized autonomous system. See below how the investments will be allocated. 

\begin{itemize}
\item \textbf{Phase 1:} 20\% of all ASR tokens will be generated.
\item \textbf{Phase 2:} 80\% of all ASR tokens will be generated.
\end{itemize}


\subsubsection{Stages}

\begin{table}[H]
\begin{tabular}{lcp{.35\textwidth}r}
  TGE STAGE & TOKENS QTY. & DATE & BONUS\\\hline
  Pre TGE & 10 Million & 19\textsuperscript{th} Feb - 12\textsuperscript{th} Mar 2019 & 50\%\\
  TGE ROUND-1 &	35 Million & 13\textsuperscript{th} Sep - 13\textsuperscript{th} Oct 2019 & 25\%\\
  TGE ROUND-2 &	- & 14\textsuperscript{th} Oct - 24\textsuperscript{th} Nov 2019 & 15\%\\
  TGE ROUND-3 &	- & 25\textsuperscript{th} Nov - 25\textsuperscript{th} Dec 2019 & 0\% \\

\end{tabular}
\caption{\label{tab:table-name}Token stages}
\end{table}


\subsubsection{Phase 1}

In the first phase 20\% of all ASR token will be distributed.

\begin{table}[H]
\begin{tabular}{llp{.62\textwidth}l}
  10\% & Public Pre-Sale & Contributions will be used to develop the network minimal viable product, and to build bigger community.\\
  5\% & Family \& Friends & Family and Friends receive their tokens as part of their compensation package.\\
  5\% & Bounty & Asure provides compensation for a number of tasks spread across marketing, bug reporting or even improving aspects of the Asure network, blockchain and platform.
\end{tabular}
\caption{\label{tab:table-name} Phase 1 - Token allocation}
\end{table}

\subsubsection{Phase 2}

In the second phase 80\% of all ASR tokens will be distributed.

\begin{table}[H]
\begin{tabular}{llp{.62\textwidth}l}
  35\% & Public Main-Sale & Contributions will be used to develop the platform, and to fund security, legal and operational needs. \\
  35\% & Foundation & Comprises foundation development and education initiatives, incentives to developers and to research blockchain, scaling, network, and platform.\\
  8\% & Team  & These are placed to acknowledge the time, effort and resources contributed to the Asure platform.  The Asure team receive their tokens as part of their compensation package, and team tokens will be vested.\\
  2\% & Advisors & Advisors receive their tokens as part of their compensation package.
\end{tabular}
\caption{\label{tab:table-name} Phase 2 - Token allocation}
\end{table}

\subsubsection{Vesting}
According to best practice and in order to protect investors and future participants of our platform, we will lock up our team’s tokens. The Asure Team will receive their tokens in twelve equal parts over two years.
The vesting ensures token course stability and commitment of all involved team members. If a holder attempts to transfer more ASR tokens than vested, the transaction will be blocked.
We are going to publish the smart contract to control vesting within our project. Hence, we will prove to the community our long-term commitment. 

\subsection{Funds allocation}

We envision that all ETH derived from the sale of Asure tokens will be allocated in the following manner:
\newline\newline

\begin{table}[H]
\begin{tabular}{llp{.6\textwidth}l}
  45\% & Platform R\&D & The creation of ongoing development of our Layer-2 network \\
  30\% & Marketing \& Operations & Additional staff and resources to cover day-to-day operations and prudent management as the organization expands. \\
  10\% & Legal & We are acutely aware of the need for rigorous compliance. We will need our own well-resourced legal support. Our principal concern is to fit within complex regulatory frameworks across the globe in order to make the growth of the community legally secure. \\
  10\% & Tax & Tax and organization development fees.\\
  5\% & Office Expenses & Office expenses and HR activities to build up
        a team to achieve roadmap goals
\end{tabular}
\caption{\label{tab:table-name}Funds allocation}
\end{table}

\subsection{KYC/AML}

The primary objective of token sale registration is to enforce a mandatory Know-Your-Customer (KYC) check to prevent identity theft, terrorist financing, Anti-money laundering (AML), and financial fraud. It also allows our team to understand our token holders better and manage risks appropriately.

The Asure tokens are not being offered or distributed to, as well as cannot be resold or otherwise alienated by their holders to citizens of, natural and legal persons, having their habitual residence, location or their seat of incorporation in the country or territory where transactions with digital coins are prohibited or in any manner restricted by applicable laws or regulations, or will become prohibited or restricted at any time after this agreement becomes effective (“Restricted Persons”).

We do not accept participation from the restricted persons and reserve the right to refuse or cancel the ASR token purchase requests at any time at our sole discretion when the information provided by the purchasers within the KYC procedure is not sufficient, inaccurate or misleading, or the purchaser is deemed to be a restricted person.

\subsection{Privacy and Security}
The security of your data is of great importance to us. There is no “cutting corners” when it comes to security, even under the pressure of running an ICO.  As such, please find below the measures which will be employed to ensure your privacy and security: 

All your data will be stored in an encrypted form on our servers
We don’t store your password as we only support external authentication providers like Google and Facebook
All the information required for the KYC process will be wiped out from our systems once the checks are completed

Asure will never share members’ personal data with 3rd parties without prior consent. In order to be on the safe side you should take these precautions: 

Never send any fiat money or crypto coins to any address during the registration process. There is only one public token sale date and it is specified on our 
website: https://www.asure.network
Bookmark the registration website, and never get to it following any email links.
Never trust emails related to the particular sale details (such as the information about soft or hard caps, Ethereum address to send to, etc.). Remember that sender’s email address can be easily forged. 
Never reply to our emails. Perform all your operations on our website only. You can check your registration status on our website using your account details. 


\subsection{Excluded participants}
Due to legal restrictions citizens and residents from the following countries are not eligible to acquire ASR tokens: American Samoa, Belarus, Burundi, Central African Republic, Cuba, Congo (Brazzaville), Congo (Kinshasa), Guam, Iraq, Iran, Lebanon, Libya, Northern Mariana Islands, North Korea, Puerto Rico, Somalia, Sudan, South Sudan, Syria, United States, US Virgin Islands, US Minor Outlying Islands, Venezuela, Yemen, Zimbabwe.


\section{Past Work}

Asure’s primary focus within the social security field is on pension insurance. As part of the ongoing research, we have ported the specific aspects of the German pension system to Ethereum blockchain. Based on both, our hands-on experience and our expertise from years of working in the insurance field, we developed the theoretical backbone of how a decentralized pension system is supposed to function as well as the proof-of-concept implementation of such a system. 

\subsection{Research on the blockchain technology and automation}

Asure’s CTO, Fabian Raetz, did a research project at the University of Applied Science and Art Dortmund in 2013 where he analyzed the emerging blockchain technologies and its possible applications. \cite{fraetz}
\newline

In 2014 a small team led by Paul Mizel and Fabian Raetz developed their own blockchain based currency as a proof of concept and tested different kinds of blockchain issues and economic systems (NRJ Coin). \cite{nrjcoin}
\newline

Paul Mizel has built a team in Kiev late 2015 for AI-based innovation projects  “Insure Chat”, “Insure Assistant” and “Insure Advisor”. The applications that were built as a result were fully automated chatbots for support, claim management, and other tasks with a unique learning mechanism and connection to social platforms like Facebook, Telegram, Skype, and others.\newline
Tech stack: IBM Watson, Microsoft Bot Framework, MS Luis, .NET.
\newline
Algorithms used: Text mining, regression analysis, SVMs, neural networks.

\subsection{German Pension System}
In order to demonstrate the potential of blockchain-based social security, Asure created a prototype based on the model of the German statutory pay-as-you-go pension system.
\newline\newline

The Asure dApp will become the reference implementation for dApps using the Asure blockchain and platform.
\newline\newline
It will feature
\begin{itemize}
\item a technical feasibility study of the german statutory pension system implemented on the Ethereum blockchain and the Asure protocol / platform.
\item a complete wallet implementation.
\item an overview and management of your insurance policies.
\item an insurance store to find and buy insurance policies.
\end{itemize}

Please try out the Asure dApp which runs currently on the Ethereum Rinkiby testnet: 
https://dapp.asure.io

\subsection{Decentralized Pension System}

To demonstrate that blockchain can solve problems globally, Asure also developed a prototype of a global pension system which is fully decentralized and hence lies neither in the hands of governments nor of any insurance company.

This is an alpha-phase experiment designed to show how social security systems can be improved in the future with the help of blockchain technology.

The idea is to implement a pay-as-you-go pension system on Ethereum blockchain. Members pay their contributions in ETH and receive ERC20 tokens in return. No contributions are invested in the capital market and therefore no interest is earned. Instead, the paid-in ETHs are used directly for the payment of outstanding pension claims. How much pension is going to be paid out depends on how many pension tokens a pensioner has, i.e. how many contributions he paid into the system.

As a rule, pay-as-you-go systems only work because states introduce mandatory social security systems and, thus can guarantee a stable number of members and contribution payments. In a decentralized pension system nobody can be forced to become a member. Asure's membership creates several incentives that are intended to lead to mass acceptance.

In the decentralized pension system as well as in a classic one, whoever makes a higher contribution gets a higher pension. Pay-ins longevity plays a role as well. The longer one makes regular pay-ins the longer the pension is going to be paid out.

\begin{figure}[H]
    \centering
    \includegraphics[width=5.0in]{img/pension.png}
    \caption{PAYG Model}
    \label{fig:payg}
\end{figure}

The Asure decentralized pension dApp runs currently on the Ethereum Rinkeby testnet. It was developed during ETHBerlin hackathon and can be accessed via the following link: 
\url{https://ethberlin.asure.io}

Pension is a bet that the value I pay in is at least as great, if not greater, as the payout. The decentralized pension is based on the German pension system and has implemented a “generation contract”. The young generation pays the older generation according to their possibilities and in return, the pension entitlements are tokenized, In the form of pension entitlement tokens (PET).
\newline\newline

\subsubsection*{Incentive models were developed within the project}
The system excludes the administration of age, thereby avoiding fraud and evidence. The time is divided into periods where a period is a month. Within each period deposits can be made. For each period a target price is fixed, which can shift if the median of the deposits of the previous period has a big difference to the target price. 

If the maximum number of periods has been paid in, the maximum number of pension payments is also possible. Let's assume that the maximum number of periods is 480 equal 40 years. For monthly payments of 40 years, there is a claim to 40 years pension. If someone has only used the system for 2 years, the application is for 1 month only. The incentive to use the system to the maximum rewards the participants with more pension entitlement period.

\begin{eqnarray}
	entitlementMonths = \frac{payedMonths^2}{12 \cdot 40 years}
\end{eqnarray}

\begin{figure}[H]
    \centering
    \includegraphics[width=3.0in]{img/pension_years.png}
    \caption{Decentralized pension payed vs. recive years}
    \label{fig:pension_years}
\end{figure}

Since everyone can pay in different amounts in the system, the maximum payer is granted a maximum of double pension entitlement. All those who pay in more than the target price of the period will receive more PET up to a maximum of 2 per period. Maximum achievable 960 PET, this allows a later claim to twice as much in redistribution as someone who activates 480 PET.

\begin{eqnarray}
	DPT = \begin{cases} 1 + \frac{amount-amount_{max}}
	{targetPrice - amount_{max}} 
	* DTP_{bonus} & amount \geq targetPrice\\
	\frac{amount - amount_{min}}
	{targetPrice - amount_{min}} 
	* DTP_{bonus} & otherwise\end{cases}
\end{eqnarray}

\begin{eqnarray}
targetPrice - amount_{max} \neq 0 \quad and \quad targetPrice - amount_{min} \neq 0
\end{eqnarray}

As a further incentive for the early adopters, a bonus was provided in the system which has a multiplicator of 1.5 and with the time logarithmically approaching 1.0 is planned to approach annually.

\begin{eqnarray}
	DTP_{bonus} = f(year) = 1.5-0.12 * log(year)
\end{eqnarray}

\begin{figure}[H]
    \centering
    \includegraphics[width=3.0in]{img/pension_bonus.png}
    \caption{Decentralized pension bonus by year}
    \label{fig:pension_bonus}
\end{figure}

If everyone leaves the system, the last participants are rewarded more, thus we guarantee that the system remains lucrative, with zero participants in the system the system is set to its initial state again.

By the limitation on maximally 2 PET or with the factor 1.5 initially 3 PET per period in the first years a utilization possibility results with several accounts into the system to pay in which the system prevents that the PETs are not transferable. 

With the help of these incentives and transparent design and DAO approach, this will start as a social experiment after necessary simulations and parameter adjustments on Ethereum mainnet.

\subsubsection*{Benefits}
Independent Crypto Pension has many advantages, the intergenerational contract allows the inflation security. It is autonomous and decentralized according to the idea of the DAO. There is no intermediary.  The privacy is secured because no personal data is necessary to participate in the system.  It is completely transparent as all transactions are on the blockchain and it is also open source.

\subsubsection*{Read more}
We summarized our ideas on how a redistribution based peer-to-peer pension system might look and share our results with the broader community.
\newline
Depot Paper: \url{https://www.asure.network/asure.depot.en.pdf}



\newpage

\section{Дальнейшея работа}

Эта работа представляет собой единый путь к созданию сети Asure; Однако мы также считаем, что эта работа станет отправной точкой для будущих исследований децентрализованных систем социального обеспечения. В этом разделе мы определяем и заполняем две категории будущей работы. Это включает в себя работу, которая была завершена и просто ожидает описания и публикации и открытых вопросов для улучшения существующих протоколов.

\subsection{Текущая работа}

Следующие темы представляют текущую работу.

\begin{itemize}
\item Реализация Plasma MVP.
\item Мобильное приложение (Android, iOS)
\item Исследование децентрализованной системы социального обеспечения.
\item Контракты и протоколы интерфейса Asure-in-Ethereum.
\item Полная реализуемая спецификация протокола Asure.
\end{itemize}

\subsection{Открытые вопросы}
Есть еще области для улучшений, которые могут положительно повлиять на производительность сети. К ним можно вернуться позже после сбора достаточного количества статистических данных, по которым можно определить важность и необходимость внесения изменений:

\begin{itemize}
\item Лучшее решение для массовых стратегий входа и выхода.
\item Безопасное решение проблемы недоступности данных.
\item Более практическое применение SNARK/STARK.
\item Лучшая стратегия для более быстрого внедрения систем социального обеспечения и новых экономических моделей.
\item Лучший примитив для функции доказательства Proof-of-Stake, которая является публично-переменной и прозрачной.
\end{itemize}

Поскольку социальное обеспечение является лишь специализированной формой страхования, очевидно, что поддержка децентрализованного страхования на платформе также очевидна, и это хорошая пара для расширения этой платформы для рынка. Экосистема Asure состоит из сети Asure, протокола Asure, платформы Asure, на которой работают потенциальные сторонние приложения в области социального обеспечения и страховой среды. Признание экосистемы будет неуклонно расти из-за возникающих сетевых эффектов и синергии. 

\section{Организация}
Asure является некоммерческой организацией, основанной на трех основных принципах: инновации, сотрудничество и исследования с сообществом участников, занимающихся исследованиями и разработками для новых разработанных решений, созданных в сети Asure, блокчейном и платформой для разработки решений блокчейна с системами социального обеспечения и страхования в стиле DAO.
\newline

Организация включает в себя исследователей технологий, а также экспертов по страхованию. Asure является неотъемлемым компонентом нашей работы, который позволяет нам координировать взаимодействие в различных частях экосистемы.

\section{Acknowledgements}

This work is the cumulative effort of multiple individuals within the Asure Foundation team, and would not have been possible without the help, comments, and review of the collaborators and advisors of Asure Foundation. We also thank all of our collaborators and advisor for useful conversations; in particular Andrey Kuchaev, Alexander Böhner, Dirk Mattern, Dennis Rittinghof, Michael Lurz, Emanuel Kuceradis and Prof. Dr. Hirsch.


\section{Zusammenfassung}

Diese Arbeit verfolgt eine Vision wo die aufgeführten Probleme mit einem dezentralem Ansatz zu lösen als auch einen Weg aufzuzeigen wie ein Sozialversicherungssystem in der Zukunft funktionieren kann. 

Durch die freiwillige Teilnahme wurden Anreize entwickelt durch die das System als eine sehr interessante alternative anzusehen ist. Ein Anfang für eine transparente, faire und eine barrierefreie Altersvorsorge die Weltweit nutzbar ist.


%Problem - Lösungsansätze.
%der demografische wandel durch globalen und dezentralen ansatz lösen lässt.
%wo wir statt stable coin mechanismen uns nicht an den doller anhängen sondern an einen %referenzwert der eine kaufkraft wiederspiegelt.
%und können wir es so designen, dass menschen auch etwas mehr in der zukunft erhalten als %diese einbezahlt haben?


Halten sich alle an die gleichen regeln, wird lediglich, das System die Kaufkraft wieder geben. Bei einer 0\% Inflation und 0\% Deflation wird die selbe Anzahl an $Units$ die in das System eingezahlt wurden an die Nutzer ausgezahlt. Das bedeutet es gibt kein Gewinn und kein Verlust, sondern der Werterhalt wurde über Jahre einfach aufbewahrt. \footnote{ Es werden lediglich die Transaktionskosten anfallen, bei eigenem Netzwerk können diese auch noch weiter minimiert werden.}

Es wird davon ausgegangen, dass einige Nutzer sich für eine frühere Rente entscheiden werden, auch wenn es Teilverlust für einen Nutzer bedeutet. Einige Nutzer werden vollständigen Zugang zum System verlieren, weil z.B. der private Schlüssel verloren geht, die Rentenbeiträge werden nicht vererbt und nicht abgeholt. Nach einiger Zeit werden die eingezahlten Werte freigegeben und dadurch werden die anderen Nutzer im System profitieren.





\newpage

\listoftables

\listoffigures
 
\newpage
% Literaturliste endgueltig anzeigen

% NEW SITE ---------------------------------------------------------------------------------
\begin{thebibliography}{9}

  \bibitem{cammarden}
  (Carmela Troncoso, Marios Isaakidis,George Danezis, Harry Halpin),
  \textit{Systematizing Decentralization and Privacy: Lessons from 15 Years of Research and Deployments, In Proceedings on Privacy Enhancing Technologies},
  De Gruyter Open,
  volume 2017,
  2017.
  
  \bibitem{plasma}
  Joseph Poon and Vitalik Buterin, Plasma: Scalable Autonomous Smart Contracts,
  \url{https://plasma.io/},
  2017.
  
 \bibitem{plasmamvp}
  Minimal Viable Plasma,
  \url{https://ethresear.ch/t/minimal-viable-plasma/426},
  2017.
  
  \bibitem{plasmacash}
  Plasma Cash,
  \url{https://ethresear.ch/t/plasma-cash-plasma-with-much-less-per-user-data-checking/1298},
  2017.
  
  \bibitem{ethereum}
  Ethereum,
  \url{https://ethereum.org},
  2014.

  \bibitem{erd}
  European Report on Development (ERD): Deutsches Institut für Entwicklungspolitik,
  \url{https://www.die-gdi.de/erd/},
  2018.
  
  \bibitem{hcms}
  Health as Human Capital: Theory and Implications A New Management Paradigm, HCMS Group,
  \url{http://www.hcmsgroup.com/wp-content/uploads/2012/05/WP01-HHC-Theory-and-Implications-2012-01-161.pdf},
  2012.
 
  \bibitem{gaslimit}
  etherscan.io: gaslimit chart,
  \url{https://etherscan.io/chart/gaslimit},
  2012. 

\end{thebibliography}


\bibliographystyle{ieeetr}
%\bibliography{literatur}



\vskip 2.2in
\begin{quote}
	\centering
	Сделано в Германии с \ensuremath\heartsuit{ } 
\end{quote}

\end{document}
