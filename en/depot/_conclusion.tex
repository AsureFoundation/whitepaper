\section{Conclusion}

This paper proposes the development of a decentralized, globally available pension system using the redistribution scheme and to use a public blockchain infrastructure to 
operate it.

We showed that the redistribution of contributions and pension payouts can be implemented without depending on personal information of the insured person (e.g. age, identity, etc.). 
%Traditionally, redistribution schemes work best if they are mandatory as it ensures steady contributions and therefore pension payouts. 
Also, we proposed a redistribution scheme with voluntary participation and proper incentivation to guarantee a steady number of new contributors. Contributors have an incentive to participate in the pension system as they are rewarded with a higher pension if they play by the rules of the pension system. 
Due to its decentralized design, the pension system is not controlled by anybody, is globally available, and open for everyone to participate. The rules of the proposed pension system are immutable and transparently saved onto the blockchain. 

The pension system stores purchasing power through the use of pension points which represent the amount of a contribution relative to all other contributions. Pension payouts are based on the amount of pension points a pensioner has and result in a pension that is worth the purchasing power a pensioner was willing to contribute. Therefore there is no profit and no loss and the value retention was being kept over the  years. \footnote{ The only costs incurred will be the transaction costs, in case of a private network, these can be further minimized.}
 
Due to its global availability, the proposed pension system can provide a pension system to people that currently don't have access to transparent and fair pension systems. Also, it could provide a useful addition to existing pension systems in a portfolio as it is tied to a different class of risks and therefore provides enhanced risk diversification.

We believe that the blockchain technology will play an important role in the design of future pension systems and we lay the first step towards that future with the proposed pension system.