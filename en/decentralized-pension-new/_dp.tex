\section{Dezentrale Rente}
Die dezentrale Rente basiert im Kern auf dem Umlageverfahren und dem
Leistungsprinzip. Je länger ein Beitragszahler in das Rentensystem einzahlt, 
desto länger werden Leistungen im Alter ausgezahlt und je höher die Beiträge,
desto höher die Rentenzahlungen im Alter.

\paragraph{Generische Währungseinheit.} Das Design der Rente ist nicht auf eine bestimmte 
Währungseinheit wie z.B. ETH oder BTC festgelegt. Stattdessen verwendet das Design des
Rentensystems die generische Währungseinheit "\textbf{unit}". Diese muss bei der konkreten
Implementierung durch ein konkrete Währung ersetzt wird.

\paragraph{Perioden.} Das Rentensytem ist in Perioden unterteilt, welche definiert werden als $P$. Jede Aktion innerhalb des Rentensystems erfolgt
in einer Periode $P[p]$. Die Dauer einer Periode ($P_{duration}$) ist frei wählbar. Innerhalb der folgenden Betrachtung gilt $P_{duration} = 1 month$.

\paragraph{Benutzer.} Alle Benutzer des Rentensystems werden als $U$ und individuelle Benutzer als $U[u]$ definiert. Jeder Nutzer hat einen Zustand $U[u]_{state}$ welcher die Zustände \textbf{UC} (Contributor),
\textbf{UP} (Pensioner) und \textbf{UD} (Done) annehmen kann.
Der den Anfangszustand aller Nutzer ist $U[u]_{state} = UC$ 

\paragraph{Konten.} Die Dezentrale Rente verfügt über zwei Konten: Rücklagen ($W_{savings}$) und Nachzügler ($W_{laggarts}$). Beitragszahlungen und Rentenzahlungen werden mithilfe dieser verrechnet. Das Rücklagenkonto enthält den Gesamtbetrag der verwalteten Beiträge und dient der monatlichen Abwicklung der Ein-, und Auszahlungen. Das Nachzüglerkonto  
verwaltet eine Rücklage, welche an die letzte Generation des Rentensystems ausgezahlt wird.

\subsection{Beitragszahlung}

Beitragszahler können in einer Periode $P[p]$ Beiträge zahlen. Der Gesamtbetrag pro Periode und Nutzer wird als $U[u]_{units[p]}$ definiert. Die Gesamtanzahl an Perioden, in denen ein Beitragszahler gezahlt hat, wird als $U[u]_{contrib}$ definiert.

\begin{equation}
U[u]_{contrib} = \sum_{p=0}^{|P|} \begin{cases} 
1 & _{if U[u]_{units[p]} > 0} \\
0 & _{otherwise}
\end{cases}
\end{equation}


Alle Beitragszahlungen einer Periode $P[p]$ werden dem Rücklagenkonto  $W_{savings}$ gutgeschrieben.
Beiträge von Beitragszahlern, welche noch über keine Rentenanspruchsperioden ($U[u]_{pensionperiods} = 0$) verfügen, werden zusätzlich auf dem Nachzüglerkonto $W_{laggarts}$ verbucht.

\begin{equation}
W_{savings} = W_{savings} + \sum_{u=0}^{|U|} U[u]_{units[p]}
\end{equation}


\begin{dmath}
W_{laggarts} = W_{laggarts} + \sum_{u=0}^{|U|} \begin{cases} 
U[u]_{units[p]} & _{if U[u]_{pensionperiods} = 0} \\
0 & _{otherwise}
\end{cases}
\end{dmath}

\subsubsection*{Rentenanspruchsperioden}
Die Rentenanspruchsperioden ($U[u]_{pensionperiods}$) eines Rentners legt die Anzahl der Perioden fest,
in denen eine Rentenzahlung erfolgt. Je höher die Anzahl der Beitragsperioden
$U[u]_{contrib}$, desto höher ist die Anzahl der Rentenanspruchsperioden.

Um eine hohe Anzahl an Beitragsperioden zu icentivieren, wird ein Zielwert ($P_{target}$)
für die Anzahl der Beitragsperioden definiert. 

\begin{equation}
	P_{target} = 40 years \cdot 12 months
\end{equation}

Wenn die Anzahl an Beitragsperioden dem Zielwert entspricht, soll die Anzahl der 
Rentenanspruchsperioden genau dem Zielwert entsprechen. Liegt die  Anzahl an
Beitragsperioden unter oder über dem Zielwert, soll die Anzahl der 
Rentenanspruchsperioden entsprechend überproportional kleiner oder größer ausfallen.  

Die Anzahl der Rentenanspruchsperioden ist wie folgt definiert:

\begin{equation}
U[u]_{pensionperiods} = \frac{U[u]_{contrib}^2}{P_{target}}
\end{equation}


\subsubsection*{Dezentrale Rentenpunkte}
Für die gezahlten Beiträge einer Periode ($U[u]_{units[p]}$) erhält ein Beitragszahler Dezentrale Rentenpunkte
($DPT$). Aus der Gesamtanzahl der DPT eines Beitragszahlers wird im Rentenalter
die Höhe der zu zahlenden Rente errechnet.

\begin{equation}
U[u]_{dpt[p]} = DPT(u, p)
\end{equation}

\begin{equation}
DPT(u, p) = DPT_{base}(u, p) \cdot DPT_{bonus}(p)
\end{equation}

\paragraph*{Aktueller Beitragswert}
Der aktuelle Beitragswert (CCV) wird zu Beginn jeder Periode 
$P[n]$ berechnet und als $P[p]_{ccv}$ definiert. Der CCV bildet den Referenzwert für einen $DPT$ der
entsprechenden Periode.

Zur Berechnung des CCV der Periode $P[p]$ wird der CCV der vorherigen
Periode $P[n-1]$ zugrunde gelegt. Ist die Differenz zwischen CCV und dem
durchschnittlichen Beitrag der Periode $P[n-1]$ größer als 10\%, so wird
der CCV der der Periode $P[n]$ entsprechend um 10\% erhöht oder gesenkt.\\
Bei starken Schwankungen des durchschnittlichen Beitrags, nährt sich der CCV langsam an den neuen Durchschnitt an und große Sprünge werden vermieden.


\begin{equation}
P[p]_{units} = avg(\sum_{u=0}^{|U|} U[u]_{units[p]})
\end{equation}

\begin{equation}
P[p]_{ccv} = \begin{cases} 
P[p]_{ccv} \cdot 1.1 & _{if P[p]_{units} \cdot 1.1 > P[p-1]_{ccv}} \\
P[p]_{ccv} \cdot 0.9 & _{if P[p]_{units} \cdot 0.9 < P[p-1]_{ccv}} \\
P[p]_{ccv} & _{otherwise}
\end{cases}
\end{equation}

\paragraph*{Dezentrale Rentenpunkte Basis}
Die Rentenpunkte ist die ERC20 Tokensierung der Kaufkraft in Blockchain-Systemen und ist eine Abstraktion zur Kaufkraft wo der $CCV$-Wert die Referenz abbildet wie die Bereitschaft zur Zahlung in den Vergangenen Perioden vorhanden war. Bei einer Einzahlung die gleich dem CCV-Wert entspricht werden 1.0 DPT Punkte dem Sender gutgeschrieben. Alles über dem CCV Wert werden Maximal 2.0 DPT gutgeschrieben werden, wobei es möglich ist mehr als das doppelte dem CCV-Wert zu bezahlen. Bei einem kleineren Beitrag als dem CCV-Wert wird dem entsprechend Proportional zum min weniger DPT gutgeschrieben werden.

\begin{dmath}
DPT_{base}(u, p) = \begin{cases} 
min(\frac{U[u]_{units[p]}} {P[p]_{ccv}}, 2) 
  & _{if U[u]_{units[p]} > P[p]_{ccv}} \\
\frac{U[u]_{units[p]} - min} {P[p]_{ccv} - min} 
  & _{if U[u]_{units[p]} < P[p]_{ccv}} \\
1.0 & _{otherwise}
\end{cases}
\end{dmath}

\paragraph*{Dezentrale Rentenpunkte Bonus}
Bis zur Periode $P_{bonus} = 480$ werden zusätzliche Bonus DPT 
($DPT_{bonus}$) an
Beitragszahler ausgegeben. Bonus DPT sollen einen Anreiz für
die ersten Nutzer des Rentensystems schaffen und so die Nutzer belohnen, 
welche früh an das System geglaubt und investiert haben. Das Konzept der Bonus DPT
ist an die aus dem Bitcoin Protokoll bekannte Reduzierung der Mining Rewards angelehnt.

\begin{equation}
DPT_{bonus}(p) = \begin{cases} 
1 + \frac{(P_{bonus} - P[p] + 1)^2}
      {2 \cdot P_{bonus}^2} & _{if P[p] < P_{bonus}} \\
1 & _{otherwise} 
\end{cases}
\end{equation}


\subsection{Rentenzahlung}

Beitragszahler können frei wählen, wann Sie in Rente gehen. Geht ein Beitragszahler in Rente, können keine Beitragszahlungen mehr getätigt werden und je nach gezahlten Beiträgen, können stattdessen Rentenauszahlungen erfolgen. Der Übergang von Beitragszahler zu Renter wird durch die Änderung des Zustandes $U[u]_{state} = UP$ makiert.

Die Berechnung der zu zahlenden Rente erfolgt anhand der Gesamtanzahl der DPT eines Rentners ($U[u]_{dpt\_total}$).

\begin{equation}
U[u]_{dpt\_total} = \sum_{p=0}^{|P|} U[u]_{dpt[p]}
\end{equation}

Die Rentenanspruchsperioden $U[u]_{pensionperiods}$ legen fest, in wie vielen Perioden eine Rente ausgezahlt wird. In
welchen Perioden ein Rentner seine Rentenzahlungen in Anspruch nimmt, bleibt diesem selbst überlassen und kann frei
gewählt werden. Wurden alle Rentenzahlungen in Anspruch genommen, wird der Zustand ensprechend durch $U[u]_{state} = UD$ ersetzt.

Hat ein
Rentner in seinen Beitragsperioden immer den durchschnittlichen Beitrag in das
Rentensystem eingezahlt, soll seine
Rentezahlung auch den durchschnittlichen Beitragszahlungen der aktuellen
Periode entsprechen und so die in $DPT$ gespeicherte Kaufkraft wieder hergestellt werden.


Die Dezentrale Rente verfügt über drei Geldtöpfe (Beiträge, Rücklagen und Bonus) 
aus denen die Rente zusammengesetzt und ausgezahlt wird. Die Beiträge werden wie folgt auf 
die drei Geldtöpfe verteilt:
\begin{compactenum}
\item Beiträge von Beitragszahlern, welche noch über keine Rentenanspruchsperioden ($U[u]_{pensionperiods} = 0$) verfügen, werden in den Bonustopf abgelegt.
\item Beiträge von Beitragszahlern, welche über Rentenanspruchsperioden ($U[u]_{pensionperiods} > 0$) verfügen, werden in dem Beitragstopf abgelegt.
\item Der Restwert des Beitragstopf wird nach Auszahlung der Renten einer Periode in den als Rücklage verwendet.
\end{compactenum}

Für jeden Geldtopf wird eine Umrechnungsrate berechnet, welche pro 
Periode festlegt, wie viel ein $DPT$ aus dem entsprechenden
Geldtopf Wert ist.

Die Rentenzahlung wird somit wie foglt definiert:

\begin{equation}
U[u]_{pension[p]} = U[u]_{dpt\_total} \cdot (CPR(p) + SPR + LPR)
\end{equation}

\subsubsection*{Beitragsrentenrate ($CPR$)}
Alle Beiträge einer Beitragsperiode ($P[p]units$) werden gesammelt und anteilig an Rentner ausgezahlt. Die Beitragsrentenrate $CPR(p)$ definiert die Umrechnungsrate von DPT zu Beiträgen.

\begin{equation}
P[p]_{units} = \sum_{u=0}^{|U|} U[u]_{units[p]}
\end{equation}

Die zu zahlenden Renten sind durch die durchschnittliche Beitragszahlung der 
jeweiligen Periode ($avg(P[p]units)$) gedeckelt. Gibt es mehr Beitragszahler als Rentner im System, werden so eventuelle Überschüsse als Rücklage verwendet und nicht direkt ausgezahlt. Reichen die Beitragszahlungen nicht, um die durchschnittliche Beitragszahlung der 
jeweiligen Periode zu zahlen, werden die Beiträge proportional anhand der DPT auf alle Rentner und Rentenzahlungen  aufgeteilt.

\begin{equation}
P[p]_{dpt\_pensioner} = 
\sum_{u=0}^{|U|} \begin{cases} 
U[u]_{dpt\_total[p]} & _{if U[u]_{state} = UP}\\
0 & _{otherwise}
\end{cases}
\end{equation}

\begin{equation*}
CPR_{a}(p) = \frac{avg(P[p]_{units})}{P_{target}}
\end{equation*}

\begin{equation*}
CPR_{b}(p) = \frac{P[p]_{units}} {P[p]_{dpt\_pensioner} \cdot avg(P[p]_{units}))}
\end{equation*}

\begin{equation}
CPR(p) = min(CPR_{a}(p), CPR_{b}(p))
\end{equation}


\subsubsection*{Rücklagenrentenrate ($SPR$)}

% auc = active user count

\begin{equation}
P[p]_{auc} = \sum_{u=0}^{|U|} \begin{cases} 
0 & _{if U[u]_{state} = DP}\\
1 & _{otherwise}
\end{cases}
\end{equation}

\begin{equation}
P[p]_{active\_dpt} = \sum_{u=0}^{|U|} \begin{cases} 
0 & _{if U[u]_{state} = DP}\\
U[u]_{dpt\_total} & _{otherwise}
\end{cases}
\end{equation}

\begin{equation*}
TOPP(u) = \sum_{p=0}^{|P|} \begin{cases}
1 & _{if U[u]_{pension[p]} > 0}\\
0 & _{otherwise}\\
\end{cases}
\end{equation*}


% top = total open periods

\begin{equation}
P[p]_{top} = \sum_{u=0}^{|U|} \begin{cases} 
P_{target} - TOPP(u)  & _{if U[u]_{state} = DP}\\
P_{target} & _{if U[u]_{state} = DC}\\
0 & _{otherwise}
\end{cases}
\end{equation}

\begin{equation}
SPR(p) = \frac{total\_savings - laggarts\_savings} {P[p]_{active\_dpt} \dot {\frac{P[p]_{top}} {P[p]_{auc}}}
}
\end{equation}


\subsubsection*{Bonusrentenrate ($LPR$)}

Wenn $entitlementMonths >= 1$ dann 

\begin{equation*}
	active\_user\_dpts = 
	\sum_{n=1}^{count} active\_user\_dpt
\end{equation*}

\begin{equation*}
	total\_units = 	
	\sum_{p=1}^{periods} \sum_{n=1}^{users} user\_units	
\end{equation*}

\begin{equation*}
	laggards\_rate = 	
	\frac{total\_units} 
	{active\_user\_dpts \cdot years \cdot 12}		
\end{equation*}

\begin{equation*}
	pension = laggards\_rate \cdot user\_dpt
\end{equation*}


Voraussetzung ist gegeben wenn $pensioners / periods_open$ genau 1 entspricht