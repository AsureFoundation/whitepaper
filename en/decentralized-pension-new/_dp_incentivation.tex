\subsection{Rentenzahlung}
Die Berechnung der zu zahlenden Rente wird anhand der $DPT$ eines Rentners bestimmt.
Hat ein
Rentner in seinen Beitragsperioden immer den durchschnittlichen Beitrag in das
Rentensystem eingezahlt, soll seine
Rente auch dem durchschnittlichen Beitrag der aktuellen
Periode entsprechen und so die in $DPT$ gespeicherte Kaufkraft wieder hergestellt werden.


Die Dezentrale Rente verfügt über drei Geldtöpfe (Beiträge, Rücklagen und Bonus) 
aus denen die Rente zusammengesetzt und ausgezahlt wird. Die Beiträge werden wie folgt auf 
die drei Geldtöpfe verteilt:
\begin{compactenum}
\item Beiträge von Beitragszahlern, welche noch über keine Rentenanspruchsperioden ($P_{pension = 0}$) verfügen, werden in den Bonustopf abgelegt.
\item Beiträge von Beitragszahlern, welche über Rentenanspruchsperioden ($P_{pension > 0}$) verfügen, werden in dem Beitragstopf abgelegt.
\item Der Restwert des Beitragstopf wird nach Auszahlung der Renten einer Periode in den als Rücklage verwendet.
\end{compactenum}

Für jeden Geldtopf wird eine Umrechnungsrate berechnet, welche pro 
Periode festlegt, wie viel ein $DPT$ aus dem entsprechenden
Geldtopf Wert ist.

\begin{equation}
pension = DPT \cdot (CPR + SPR + LPR)
\end{equation}

\subsubsection*{Beitragsrentenrate ($CPR$)}
Alle Beiträge einer Beitragsperiode werden gesammelt und anteilig an die jeweils
aktiven Rentner ausgezahlt.
Die zu zahlenden Renten sind durch die durchschnittlichen Beitragszahlungen der 
jeweiligen Periode gedeckelt. Eventuelle Überschüsse werden als Rücklage verwendet.

\begin{equation}
	units\_month = 
	\sum_{n=1}^{count} units
\end{equation}

\begin{equation}
	dpt\_month = 
	\sum_{n=1}^{count} pension\_user\_dpt
\end{equation}

\begin{equation}
CPR = min(\frac{avg(units)}{P_{target}},\frac{units\_month} {dpt\_month \cdot avg(units)})
\end{equation}


\begin{equation}
CPR = \begin{cases} 
	\frac{avg(units)}{P_{target}} & _{(CPR > avg(units))} \\
	\frac{total\_units\_month  \cdot avg(units)} 
		{total\_dpt\_month} & _{(other)}
\end{cases}
\end{equation}



%\begin{equation}
%	pension = 	
%	weighted\_dpt\_units\_rate \cdot user\_dpt
%\end{equation}


\subsubsection*{Rücklagenrentenrate ($SPR$)}

Wenn $weighted\_dpt\_units\_rate < avg(units)$ dann 

\begin{equation}
	active\_user\_dpts = 
	\sum_{n=1}^{count} active\_user\_dpt
\end{equation}

\begin{equation}
	total\_units = 	
	\sum_{p=1}^{periods} \sum_{n=1}^{users} user\_units	
\end{equation}

\begin{equation}
	savings\_dpt\_units\_rate = 	
	\frac{total\_units} 
	{active\_user\_dpts \cdot years \cdot 12}		
\end{equation}

\begin{equation}
	pension = savings\_dpt\_units\_rate \cdot user\_dpt
\end{equation}


\subsubsection*{Bonusrentenrate ($LPR$)}

Wenn $entitlementMonths >= 1$ dann 

\begin{equation*}
	active\_user\_dpts = 
	\sum_{n=1}^{count} active\_user\_dpt
\end{equation*}

\begin{equation*}
	total\_units = 	
	\sum_{p=1}^{periods} \sum_{n=1}^{users} user\_units	
\end{equation*}

\begin{equation*}
	laggards\_rate = 	
	\frac{total\_units} 
	{active\_user\_dpts \cdot years \cdot 12}		
\end{equation*}

\begin{equation*}
	pension = laggards\_rate \cdot user\_dpt
\end{equation*}


Voraussetzung ist gegeben wenn $pensioners / periods_open$ genau 1 entspricht