\section{Дальнейшея работа}

Эта работа представляет собой единый путь к созданию сети Asure; Однако мы также считаем, что эта работа станет отправной точкой для будущих исследований децентрализованных систем социального обеспечения. В этом разделе мы определяем и заполняем две категории будущей работы. Это включает в себя работу, которая была завершена и просто ожидает описания и публикации и открытых вопросов для улучшения существующих протоколов.

\subsection{Текущая работа}

Следующие темы представляют текущую работу.

\begin{itemize}
\item Реализация Plasma MVP.
\item Мобильное приложение (Android, iOS)
\item Исследование децентрализованной системы социального обеспечения.
\item Контракты и протоколы интерфейса Asure-in-Ethereum.
\item Полная реализуемая спецификация протокола Asure.
\end{itemize}

\subsection{Открытые вопросы}
Есть еще области для улучшений, которые могут положительно повлиять на производительность сети. К ним можно вернуться позже после сбора достаточного количества статистических данных, по которым можно определить важность и необходимость внесения изменений:

\begin{itemize}
\item Лучшее решение для массовых стратегий входа и выхода.
\item Безопасное решение проблемы недоступности данных.
\item Более практическое применение SNARK/STARK.
\item Лучшая стратегия для более быстрого внедрения систем социального обеспечения и новых экономических моделей.
\item Лучший примитив для функции доказательства Proof-of-Stake, которая является публично-переменной и прозрачной.
\end{itemize}

Поскольку социальное обеспечение является лишь специализированной формой страхования, очевидно, что поддержка децентрализованного страхования на платформе также очевидна, и это хорошая пара для расширения этой платформы для рынка. Экосистема Asure состоит из сети Asure, протокола Asure, платформы Asure, на которой работают потенциальные сторонние приложения в области социального обеспечения и страховой среды. Признание экосистемы будет неуклонно расти из-за возникающих сетевых эффектов и синергии. 