\section{Einführung}
In Kooperation mit der ECHTNICE GmbH, hat die Asure Stiftung einen Prototypen des deutschen gesetzlichen Rentensystems auf der öffentlichen Ethereum Blockchain entwickelt.

\subsection{Sozialversicherungen auf Blockchain}
Digitale Währungen / Kryptowährungen\\
Transparenz\\
Verluste reduzieren\\

\subsection{Ziele}
Mit der Entwicklung eines Prototypen zum deutschen gesetzlichen Rentensystems auf Basis der öffentlichen Ethereum Blockchain verfolgt die Asure Stiftung die folgenden Ziele:

\subsection{Abgrenzung}
Der Fokus dieser Arbeit liegt darin, das bestehende System und die Machbarkeit und die Grenzen bei der Umsetzung aufzuzeigen, aus diesem Grund werden nicht alle Themen in vollem Umfang behandelt oder direkt ausgeschlossen.

\subsubsection{GDPR, DSGVO und der Datenschutz}
Anwendungen, welche auf einer Blockchain betrieben werden und personenbezogene Daten verarbeiten, sind aus den folgenden Gründen problematisch:

\quote{
Die Anonymisierung personenbezogener Daten. Es gibt intensive Debatten und
Derzeit besteht kein Konsens darüber, was zur Anonymisierung personenbezogener Daten erforderlich ist
bis zu dem Punkt, an dem die resultierende Ausgabe möglicherweise in a. gespeichert werden kann
Blockchain-Netzwerk. Zum Beispiel kann das Hashing von Daten nicht
in vielen Situationen als Anonymisierungstechnik angesehen werden,
und doch gibt es Fälle, in denen die Verwendung von Hashing einzigartig zu generieren
Digitale Signaturen von Daten, die außerhalb der Kette gespeichert werden, sind potenziell
denkbar an einer Blockchain.

Die Ausübung einiger Rechte der betroffenen Personen. Wir stellen fest, wenn persönlich
Da die Daten in einem Blockchain-Netzwerk aufgezeichnet werden, kann es schwierig sein, sie zu korrigieren
oder entfernen Sie es. Definieren, was im Kontext als Löschen betrachtet werden kann
von Blockchains wird diskutiert.
} \cite{gdpr}

Datenschutz (zk-Snark + Datensparsamkeit)

\subsubsection{Blockchain-Skalierung}
Die öffentliche Ethereum Blockchain kann zwischen 7-15 Transaktionen pro Sekunde (TPS) verarbeiten \cite{vitalikscale}. Pro Monat stehen demnach ca. 18-39 Millionen Transaktionen zur Verfügung. Auf der öffentlichen Ethereum Blockchain laufen diverse Anwendungen verschiedener Hersteller die sich dementsprechend auch gegenseitig beeinflußen. Benötigt eine Anwendung besonders viele Transaktionen, stehen diese dementsprechend nicht mehr für andere Anwendungen zur Verfügung \cite{cryptokitty}.  

2018 umfasste die deutsche Rentenversicherung ca. 37,5 Millionen aktiv Versicherte und ca. 21,04 Millionen Rentner. \cite{rente2018,rentezahlen2019} Für ein Blockchain-basiertes Rentensystem, welches für jede Ein-, und Auszahlung genau eine Transaktion benötigt, würden so ca. 58,04 Millionen Transaktionen pro Monat erforderlich sein.

Dies ist nur eine simple Betrachtung und je nach Implementierung kann die Anzahl der benötigten Transaktionen für ein Blockchain-basiertes Rentensystem für Deutschland stark reduziert werden. Dennoch wird deutlich, dass insbesondere die öffentliche Ethereum Blockchain nicht die benötigte Anzahl an Transaktionen zur Verfügung stellt.

\paragraph*{}
Das Problem der Blockchain-Skalierung ist ein bekanntes Problem zu welchem diverse Lösungsvorschläge erforscht und erarbeitet werden \cite{scaling}. Im Rahmen dieser Arbeit wird das Thema Blockchain-Skalierung nicht betrachtet.

\subsubsection{Alternative Blockchain-Projekte}
Im Rahmen dieses Projektes haben wir uns ausschließlich auf die öffentliche Ethereum Blockchain und die dazugehörige Programmiersprache Solidity fokusiert.

\paragraph*{}
Neben der öffentlichen Ethereum Blockchain, kann Ethereum auch in privaten oder konsortium basierten Szenarien verwendet werden. Der Einsatz in privaten oder konsortium basierten Szenarien kann sich u.U. stark von dem öffentlichen Szenario hinsichtlich Skalierung, Performance und dem Umgang mit personenbezogenen Daten unterscheiden und wurde explizit nicht betrachtet.

\paragraph*{}
Auch alternative Blockchain-Projekte wie z.B. 
Hyperledger Fabric\footnote{\url{https://www.hyperledger.org/projects/fabric}}, 
Qtum\footnote{\url{https://qtum.org/en}}
oder EOS\footnote{\url{https://eos.io/}}
wurden im Rahmen dieser Arbeit nicht betrachtet. 