\section{Introduction}

%\begin{itemize}
%\item Private vs Sozialeversicherung
%\item Verfügbarkeit in Ländern ohne Sozialversicherung / Globale Versicherung
%\item Inflation
%\item Instrumentalisierung Politk
%\item Korruption und Fehlinvestiotionen
%\end{itemize}
%\subsubsection*{Altervorsorge}

Die Entwicklung in den letzten 150 Jahren hat dazu geführt, dass sich die Altersvorsorge durch den Familienverband hin zu größeren Gruppen (Staat, Kollektiv der Versichertengemeinschaft) verlagert hat. Rentensysteme sind heutzutage ein wesentlicher Bestandteil der wirtschaftlichen Entwicklung von Ländern und dennoch, gibt es 4.1 Mrd. Menschen ohne Zugang zu Sozialversicherungen.\cite{ilo}

In den Industrienationen sind die Rentensysteme auf mehreren Säulen aufgebaut. Gesetzliche-, Betriebliche- und Private-Vorsorge damit wird der Altersarmut vorgebeugt. In diesem Papier beschreiben wir eine weitere Privatvorsorge, die Dezentrale Rente.

Die Dezentrale Rente soll nach dem Vorbild der Dezentralisierten Autonomen Organisation (DAO) von der Community für die Community existieren und mehrere  Probleme von heute vorhandenen Systeme beheben.

\subsection{Probleme}

\textbf{Staatliche Unterhaltung:} Die jeweils aktive und leistungsfähige Generation betreibt private Altersvorsorge und beteiligt sich zusätzlich ...
Staatlich organisierte Rentensysteme können, im Gegensatz zu privatwirtschaftlich organisierten Rentensystemen, sowohl das Kapitaldeckungsverfahren als auch das Umlageverfahren verwenden. 
Der Staat macht die Sicherungen und Versprechen eine Rente auszuzahlen doch bei einer Staatskrise und einem Finanzzusammenbruch wird auch ein Staat den Lebensstandard nicht sichern können, es funktioniert solange gut, solange die Wirtschaft eines Landes eine gewisse Stabilität besitzt.\\
\textbf{Kapitalgedeckte Rentensysteme:} Die Inflation führt bei der Vermögenbildung im Rahmen der Altersvorsorge zu einem erheblichen Wertverlust. 
Rentensystem auf Basis des Kapitaldeckungsverfahren investieren Beiträge um eine Rendite zu erzielen, welche den Wertverlust durch die Inflation ausgleicht und im besten Fall zu einer Wertsteigerung führt. \\
\textbf{Instrumentalisierung durch die Politik:} Probleme in Sozialversicherungen sind die Instrumentalisierung durch die Politik.
Die Rentensysteme werden in Wahlprogrammen gerne für die Wahlversprechen genommen um diese Anzupassen und somit auch die Wahlen zu Gewinnen, dies kann auch Nachteile haben, da eine Politik-Periode kurzlebiger ist als ein Rentenanspruch. \\
\textbf{Letzte Generation:} Bei den Umlageverfahren ist es wichtig, dass Sichergestellt wird, dass es eine nächste Generation gibt, falls dies nicht der Fall ist wird die letzte Generation am meisten im System verlieren.\\
\textbf{Demografische Wandel:} Umlageverfahren ist ein gutes Verfahren, doch in den Industrienationen kommen immer vermehrt Probleme des Demografischen Wandels, dabei gibt es immer weniger Zahler als Empfänger und dies stellt die Regierungen vor neue Herausforderungen. Die National nicht einfach langfristig gelöst werden können. \\
\textbf{Rest Risiko:} Alle Rentensysteme sind nicht risikofrei, dies liegt in der Natur von Risiko orientierten Systemen, die Lösungsansätze liegen in der Risiko Minimierung, durch verschiedene Ansätze wie Risikostreuung, Risikoübernahme durch das Land, alternative Altersvorsorge wie Immobilien und passives Einkommen kann eine Altersvorsorge gut umgesetzt sein.


\subsection{Unser Ansatz: Dezentrale Rente}

Die erste Frage die wir uns gestellt haben ist, kann eine Sozialversicherung ohne Einfluss von Firmen oder Regierungsbehörden funktionieren? Die wir nach einiger Zeit mit Ja beantwortet haben.

Durch unsere \textbf{Vorarbeit} und die Auseinander setzung mit Rentensystemen haben wir die \textbf{Anforderungen} an ein dezentralisiertes Rentenmodell erstellt, die  \textbf{Zielgruppe} definiert und das \textbf{Umlageverfahren} als Basis genommen und um die \textbf{Incentivierung} erweitert und die daraus resultierende \textbf{Vorteile} in diesem Kapitel beschrieben.

\subsubsection*{Vorarbeit}

Als Inspiration haben wir uns an dem deutschen Umlageverfahren orientiert. Nach dem wir das deutsche Rentensystem auf Ethereum in Grundzügen implementiert haben. Haben wir die Herausforderungen gesehen die es zu Lösen galt.
Neben der Herausforderungen haben wir auch Chancen gesehen das System auch zu verbessern. z.B. bei der Vererbung des Rentenanspruchs.


\subsubsection*{Anforderungen}
Wir haben uns mit Experten aus dem Versicheurngs- und Rentensystemumfeld unterhalten um ein Model zu entwickeln welches dezentral funktionieren kann.

Da wir aus geopolitischen gründen nicht möglich auf diese Voraussetzungen wert zu legen, deswegen haben wir alternativen erarbeitet.
Wir schaffen Anreize um bei der Beitragspflicht nachhelfen zu können und bei dem Beitragswert führen wir einen von Mitgliedern repräsentativen Referenzbeitragssatz der von den Mitgliedern dynamisch sich an das Verhalten anpasst.

Dezentrale Rente mittels Umlageverfahren und Incentivierung als SmartContract.
Global Verfügbar. Keine Verwaltungskosten (außer Tx Kosten) oder Provision

\textbf{GDPR\cite{gdpr} Konform:} Wir Verzichten auf alle externe Datenquellen wie die Altersangabe, Sterbeurkunde und das Durchschnittsgehalt als Referenzwert.\\
\textbf{100 Jahre Lebenszyklus:} Das System ist auf 100 Jahre Lebensdauer ausgelegt, beim 20 Jahre Start im System und 40 jähriger Zahlungsdauer. \footnote{ Es können unterschiedliche Produkte mit unterschiedlicher Laufdauer erstellt werden.} \\
\textbf{Basispunkte Limitierung auf 2.0 Punkte:} Rentenpunkte werden in der Basis auf 2.0 limitiert, dies hat den Hintergrund, dass keiner bei der Späteren Umverteilung einen unkalkulierbaren Anspruch haben darf.\\
\textbf{Vererbbar:} Die Rente ist in Voller Höhe weiter Vererbbar


\subsubsection*{Zielgruppe}
Die Zielgruppe für ein dezentralisiertes Rentensystem sind Menschen

\begin{compactenum}
\item ohne Rentenzugang
\item wo es einen Rentenzugang gibt, dieser aber 
 \begin{itemize}
 \item Korrupt ist
 \item Intransparent ist
 \item zu hohe Inflation im Land existiert
 \item zu hohe Kosten verursacht
 \item keine guten Investitionen getätigt werden
 \item kein Vertrauen existiert
 \end{itemize}
\item die als weitere Ergänzung 
 \begin{itemize}
 \item Ihr Risiko über mehrere Risikoklassen streuen
 \item ein Crypto / Dezentralisiertes Produkt nutzen wollen
 \item die als digital Nomade leben
 \item nach alternativen suchen
 \end{itemize}
\end{compactenum}

\subsubsection*{Umlageverfahren}

%\begin{itemize}
%\item Beitragszahlung und Berechnung DPT
%\item Rentenzahlung und Umverteilung der monatichen Beiträge
%\item Rentenzahlung und Umverteilung der Rücklagen
%\end{itemize}

Umlageverfahren haben große Vorteile, dass diese Schnell eingeführt werden können und kein Kapital aufgebaut werden muss.

Das Ziel von Umlageverfahren ist es die Kaufkraft über das System im ökonomischem Sinne zu speichern, dazu werden die Renten Punkte je Einzahlung als Repräsentation des Beitrags gespeichert und nicht der Beitragswert.
Beim Renteneintritt wird anhand dieser Punkte zum Referenzwert der Rentenbeitrag berechnet. 

\textbf{Beispiel:} Wird ein fester Prozentsatz von 8,6\% vom Durchschnittsgehalt in ein Rentensystem eingezahlt bekommt der Einzahler genau 1.0 Rentenpunkte.


\subsubsection*{Incentivierung}

%\begin{itemize}
%\item Beitragsjahre = Bezugsjahre
%\item Early Adopter Bonus
%\item Bonus für letzte Teilnehmer im System
%\end{itemize}

Dezentralisierte Lösungen wie die Dezentralisierte Renten können nur dank einer gut durchdachten Anreizstruktur über die Jahre hinweg organisch wachsen. Da die Nutzung einem Benutzer selbst überlassen ist, ist es vergleichbar mit Bitcoin\cite{bitcoin}, dass sich dieses System nur durch Vertrauen und die Anreize.


\subsection{Unser Beitrag}
Es gibt vielfältige Vorteile die ein dezentralisiertes Rentensystem bieten kann, auf die wesentlichen wollen wir hier eingehen.

\paragraph{Decentralized.} Mit hilfe der Blockchain technologie ist das System dezantralisiert verfügbar und dies ermöglicht einen 24/7 Zugang weltweit.

\paragraph{Autonomous.} Es sind keine Mitarbeiter Notwendig um das system betreiben zu müssen und dies reduziert enorm die Verwaltungskosten.

\paragraph{Permissionless.} Der Zugang zu diesem System ist für jeden mit Internetzugang zugänglich.

\paragraph{Transparent.} Es ist Open-Source und jeder kann die Transaktionen einsehen und das System auf seine Verarbeitung prüfen.

\paragraph{Without any intermediary.} Dieses System soll der Community zur Verfügung gestellt werden um Aufzuzeigen, dass es auch anders funktionieren kann und die Armut auf der Welt kann dadurch reduziert werden.

\paragraph{Reduce costs.} Durch die Automatisierung und Dezentralisierten Betrieb gibt es keinen weiteren Kostenaufwand außer den Transaktionsgebühren.

\paragraph{Corruption-proof.} Es gibt keine Möglichkeit das Geld entwenden zu können.

\paragraph{Tamper-proof.} Die erlaubten veränderungen des Systems sind den Mitgliedern überlassen.

\paragraph{Froud-proof.}
Betrug wird dadurch vermieden, dass wir keine Externen Informationen für den betrieb benötigen. 

\paragraph{Complete Haritable.}
Der Anspruch an eine Rente kann durch die Weitergabe des Privaten-Schlüssels übergeben werden.