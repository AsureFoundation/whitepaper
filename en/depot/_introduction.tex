\section{Introduction}

Development over the last 150 years have led to a shift in old-age provision from the family association to larger groups (state, collective of the insured community). Pension systems today are an essential part of the economic development of states and yet, there are 4.1 billion people without access to social security.\cite{noauthor_universal_2017}

There are a variety of pension systems. For instance, in Germany pension systems are categorized into the three pillars of old-age provision. The three pillars include statutory, occupational and private pension systems. Many countries use a similar classification. As a general rule, the more pension systems a person participates in, the better it is to protect against old-age poverty due to risk diversification.

\paragraph{Financing.} Occupational and private pension systems finance themselves through the funding method and generally follow the performance principle: those who contribute a lot to the pension system also get paid a lot when they get old.

Statutory pension systems finance themselves through the funding method, the pay-as-you-go method or a hybrid of the funding and the pay-as-you-go methods. In addition to the performance principle, many statutory pension insurance policies also follow the principle of solidarity. In Germany, for example, parental leave can be counted as contribution years in pension insurance.

Both the funding method and the pay-as-you-go method have proven their worth in the past. Both financing methods have their strengths and weaknesses, and opinions differ widely as to which financing method is the better one.
%Die Dezentrale Rente soll nach dem Vorbild der %Dezentralisierten Autonomen Organisation (DAO) von der %Community für die Community existieren und mehrere  Probleme %von heute vorhandenen Systeme beheben.

\subsection{Problems of old-age provision}

Good old-age provision is hard to build. In the following we will discuss some issues of old-age provision and existing pension systems.

\paragraph{Demographic change.} Life expectancy is increasing all over the world, and especially in industrialized nations, the proportion of people over the age of 60 is growing and the problem of retirement provision is becoming more pressing. The burden on pension systems, and in particular PAYG-funded pension schemes, is rising sharply as fewer contributors become available to provide pension payments. The population of developing countries will increase in the future, and thus the problem of retirement provision. For example, the United Nations estimates that by 2050, approximately two billion people will be over 60 years of age, of which as many as 80\% live in developing countries
\cite{noauthor_pensions_2009}.


\paragraph{Inflation.}  Inflation is the object of knowledge in economics, especially macroeconomics. The consumer price index (CPR) is most frequently used to measure inflation. The index is calculated with the help of a shopping basket, which is determined in a certain year (base year) representative of an average household. 
At an inflation rate of 2\%, this means from \$ 1,000 today, which in 2040 has only a purchasing power of \$ 672.97 in 2040. 
For this reason, it is important not to store the values, but to try to systematically preserve purchasing power by means of a pay-as-you-go system.

\paragraph{Mismanagement and fees.} Compared to pay-as-you-go systems, funded systems are very strongly subject to inflation. For this reason, different investment options are used, these have a higher workload and resulting administrative costs the customer must bear. Another variable is the higher volatility it increases the chance that the investments are made well as well as the risk that bad investments can be made.

%\paragraph{Entertainment.} Todo:Die jeweils aktive und leistungsfähige Generation betreibt private Altersvorsorge und beteiligt sich zusätzlich ...
%The active and efficient generation in each case operates private old-age provision and also participates ...
%State-organized pension schemes, unlike privately-organized pension schemes, can use both the funded and the pay-as-you-go method.
%The state makes the guarantees and promises to pay out a pension but in the event of a state crisis and a financial collapse even a state will not be able to secure the standard of living, it functions well as long as the economy of a country has a certain stability.

%\paragraph{Funded pension schemes.} Inflation leads to a considerable loss of value in the accumulation of assets as part of old-age provision. 
%Todo: 
%Rentensystem auf Basis des Kapitaldeckungsverfahren investieren Beiträge um eine Rendite zu erzielen, welche den Wertverlust durch die Inflation ausgleicht und im besten Fall zu einer Wertsteigerung führt.

\paragraph{Instrumentalization by politics.} Social security funds are in the hands of politicians and bureaucrats and are perfect for redistributing revenue. This circumstance allows politicians to use social insurance for electoral promises by redistributing them in favor of a group of voters, thus ensuring the next re-election. In addition, governments benefit from more money, power and prestige through social security administration. \cite{zweifel_insurance_2012}

\paragraph{Last generation.} With pay-as-you-go systems, it is important to ensure that there is a next generation, if this is not the case, the last generation will lose the most in the system as nobody is left to pay their pensions.

\paragraph{Residual risk.} 
We do not want to go into detail about other risks such as economic risk, credit risk, interest rate risk, volatility, currency risk, psychological market risk, liquidity risk, tax risks, information risk, country and transfer risk.
All pension systems are not risk-free, this is in the nature of risk-oriented systems, the solutions are based on risk minimization, through various approaches such as risk diversification, risk taking by the country, alternative pensions such as real estate and passive income, a pension plan can be well implemented.


\subsection{Our approach: Decentralized pension}

The first question we asked ourselves is, can social security work without the influence of companies or government agencies? We answered it in the affirmative.

Through our \textbf{preliminary work} and engagement with pension systems, we've created the \textbf{requirements} to a decentralized pension model that defines \textbf{target audience} and uses the \textbf{pay-as-you-go} basis and the \textbf{ Incentivation} and the resulting \textbf{benefits} are described in this chapter.

\subsubsection{Requirements}
We talked to experts from insurance and pension systems to develop a model that works decentrally.

Since geopolitical reasons make it impossible for us to attach importance to these conditions, we have developed alternative solutions.
We create incentives in order to help people to pay contributions and with the contribution value we lead a reference contribution rate representative of members which dynamically adapts to the behaviour of the members.

The most important requirement was to enable the storage of purchasing power. Another important factor is to make risk sharing in the community as fair as possible for the target groups for whom it is suitable.

\subsubsection{Target group}
The target groups for a decentralized pension system are people:

\begin{compactitem}
\item without pension access
\item where there is a pension, but
 \begin{compactitem}
 \item it is corrupt
 \item it is intransparent
 \item the country suffers from high inflation
 \item too high administrative costs
 \item no good investment strategies in the pension system
 \item less trust in the government, politics and pension system
 \end{compactitem}
\item which as a further supplement 
 \begin{compactitem}
 \item first mover, technology lover
 \item spread their risks over several risk classes
 \item want to use a decentralized (crypto) pension solutions
 \item who live as a digital nomad
 \item looking for alternatives
 \end{compactitem}
\end{compactitem}

\subsubsection{Pay-as-you-go system}


Pay-as-you-go systems have great advantages in that they can be introduced quickly and no capital needs to be built up.

The goal of pay-as-you-go is to store the purchasing power of the system in the economic sense, to the pension points per deposit are stored as a representation of the contribution and not the contribution value.
At retirement, the pension contribution is calculated on the basis of these points at the reference value\footnote{Example: In germany, it is linked to 18.6\% of the salary in 2019.}.


\subsubsection{Incentives}
Decentralized solutions such as decentralized pensions can only grow organically over the years thanks to a well thought-out incentive structure. Since the use is left to a user, it is comparable with Bitcoin\cite{nakamoto2012bitcoin}, as the system is only controlled by trust and incentives.
%TODO:dass sich dieses System nur durch Vertrauen und die Anreize.
\begin{compactitem}
 \item Early adapters bonus
 \item Pay longer, get longer
 \item Pay more, get more
 \item Laggards payout bonus
\end{compactitem}


\subsection{Our contribution}
As an inspiration we have oriented ourselves on the German pay-as-you-go system. After having implemented the German pension system on Ethereum in outline, we have seen the challenges that needed to be solved.

In addition to the challenges, we also saw opportunities to improve the system, such as the degree of automation and the creation of new incentives that are not dependent on middlemen.

There are many advantages that a decentralized pension system can offer, the most important of which we will discuss here.

\paragraph{Decentralized and Autonomous.} With the help of blockchain technology, the system is available decentralized and this allows 24/7 access worldwide. There are no employees required to operate the system and this reduces administration costs enormously.

\paragraph{Reduces costs.} Due to the automation and decentralized operation, there is no additional cost apart from the transaction fees \footnote{Except for the Tx fees, these charges may differ depending on the network used, such as Ethereum.}.

\paragraph{Transparent.} It is open-source and anyone can view the transactions and check the validity of the processes in the system.

\paragraph{Permissionless.} 
Access is available to everyone and worldwide unconditionally. Access to this system is available to anyone with Internet access.

\paragraph{Without any intermediaries.} 
There is no organization or middleman who have money access, the system is a closed economy in itself.

\paragraph{Corruption free.} There's no way we can steal the money.

\paragraph{Tamper-proof.} The permitted changes of the system are left to the members.

\paragraph{Fraud free.}
Fraud is avoided by the fact that we do not need external information for the operation. 

\paragraph{100 years life cycle.} 
The system is designed to last 100 years, with a 20-year system start and a 40-year payment period \footnote{ Different products with different running times can be created.}.  

\paragraph{Base points limit to 2.0 points.} 
Pension points are limited on the basis to 2.0, this has the background that no one may have an incalculable claim in the later redistribution.

\paragraph{Fully inheritable.}
The total pension entitlement can be inherited by passing on the private key.

\paragraph{GDPR compliant\cite{gdpr}.} 
We do not use any external data sources such as age, death certificate and average salary as reference values.

\paragraph{Incentive system.} 
Several incentives ensure the sustainability and adoption of the system in the community.


