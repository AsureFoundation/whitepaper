\section{Simulations}

Im Rahmen unserer Untersuchungen, haben wir verschiedene Simulationen des dezentralisierten Rentenmodell durchgeführt. Das Ziel ist, unterschiedliches Nutzerverhalten zu simulieren und die Anreize und das Model noch weiter zu optimieren und auf seine Tragfähigkeit hin zu überprüfen. Ein weiteres Ziel der Simulationen ist das Aufzeigen von Nutzergruppen die von dem Rentensystem profitieren oder Verluste erleiden.

Alle Simulationen wurden in der Programmiersprache Rust entwickelt und auf Github unter der MIT Lizenz veröffentlicht.\\
\url{http://github.com/AsureNetwork/asure-pension-core}

\subsection{Simulation 1: Zero-Win}

Die ersten Simulationen versucht festzustellen, dass es zu keinen Verlusten im System kommt, falls sich alle Nutzer gleich fair und konstant verhalten.

\paragraph{Sim01.} 100 Nutzer zahlen $1.0 unit$ in das Rentensystem ein und gehen gleichzeitig in die Rente.

\paragraph{Sim02.} 90 Nutzer zahlen $1.0 unit$ und 10 Nutzer Zahlen $2.0 Units$ in das Rentensystem ein und gehen gleichzeitig in die Rente.

\paragraph{Sim03.} 90 Nutzer zahlen $1.0 unit$ und 10 Nutzer Zahlen $0.1 Units$ in das Rentensystem ein und gehen gleichzeitig in die Rente.

\paragraph{Ergebnis:}

\begin{table}[hbt!]
\centering
\resizebox{\columnwidth}{!}{%
 \begin{tabular}{|c|c c c|} 
 
\hline
\multicolumn{1}{|c}{{}} &
\multicolumn{3}{|c|}{Sim01} \\ [0.5ex] 
\hline
User & Periods & Contributions & Result \\ [0.5ex] 
\hline
1..10 & 1..480 & $1.0 \cdot 480 = 480$ & 480\\ 
20..100 & 1..480 & $1.0 \cdot 480 = 480$ & 480\\ [1ex] 
 \hline
 \end{tabular}%
}
\end{table}

\begin{table}[hbt!]
\centering
\resizebox{\columnwidth}{!}{%
 \begin{tabular}{|c|c c c|} 
 
\hline
\multicolumn{1}{|c}{{}} &
\multicolumn{3}{|c|}{Sim02}\\ [0.5ex] 
\hline
User & Periods & Contributions & Result \\ [0.5ex] 
\hline
1..10 & 1..480 & $2.0 \cdot 480 = 960$ & 960\\ 
20..100 & 1..480 & $1.0 \cdot 480 = 480$ & 480 \\ [1ex] 
 \hline
 \end{tabular}%
}
\end{table}

\begin{table}[hbt!]
\centering
\resizebox{\columnwidth}{!}{%
 \begin{tabular}{|c|c c c|} 
 
\hline
\multicolumn{1}{|c}{{}} &
\multicolumn{3}{|c|}{Sim03} \\ [0.5ex] 
\hline

 User & Periods & Contributions & Result \\ [0.5ex] 
 \hline
1..10 & 1..480 & $0.1 \cdot 480 = 48$ & 48 \\ 
20..100 & 1..480 & $1.0 \cdot 480 = 480$ & 480 \\ [1ex] 
 \hline
 \end{tabular}%
}
\end{table}

Die Simulation zeigt, dass bei einem gleichen und fairem Verhalten und keinem Einfluss von Inflation und Deflation es keine Gewinner und keine Verlierer gibt, jeder bekommt seine Einzahlungen einfach zurück.

\subsection{Simulation 2: Inflation/Deflation}

In dieser Simulationsreihe wird das Verhalten von Inflation und Deflation der Units simuliert. Es ist die Entwertung und Aufwertung der Units, welches einen Einfluss auf den Benutzer hat. Bei einer Abwertung würden die Nutzer mehr Bezahlen und bei einer Aufwertung dementsprechend weniger wobei die Kaufkraft gleich bleiben würde. Durch $CCV$ wird dies automatisch aufgefangen und die Renten Ein- und Auszahlungsverhalten dementsprechend angepasst. 

\paragraph{Sim14.} Bei einer Inflation von 5\% fehlt der Wert der Units und dementsprechend Zahlen die Benutzer mehr Units. Nehmen wir 2 Generationen mit je 100 Nutzern und lassen diese bei 5\% jährlicher Inflation in das System zahlen, so zahlt die erste Generation am Anfang $1 Unit$  und am Ende $21.725 Units$. Die nächste Generation beinhaltet keine Inflation, damit das Ergebnis weniger verfälscht wird und einfacher zu Interpretieren ist. 

\paragraph{Ergebnis:}

\begin{table}[hbt!]
\centering
\resizebox{\columnwidth}{!}{%
 \begin{tabular}{|c|c c c|c c c|c c c|} 
 
\hline
\multicolumn{1}{|c}{{}} &
\multicolumn{3}{|c|}{Sim14}\\ [0.5ex] 
\hline
User & Periods & Contributions & Result \\ [0.5ex] 
\hline
1..100 & 1..480 & $\sum_{j=0}^{39} 12 \cdot 1.05^j = 1456$ & 2320 \\ 
100..200 & 480..960 & $\sum_{j=40}^{79} 12 \cdot 1.05^{j} = 3380$ & 2514 \\ [1ex] 
 \hline
 \end{tabular}%
}
\end{table}

Bei diesem Ergebnis sehen wir, dass bei einer Inflation die Units entwertet werden und in beim bestehen des Inflationnevous erhält der Nutzer aus dem System zur Kaufkraft passende Anzahl an Units.

\paragraph{Sim15.} Bei einer Deflation, wo der Wert der Units steigt wird der Benutzer dementsprechend weniger in ein System zahlen. Nehmen wir 2 Generationen mit je 100 Nutzern und lassen diese bei 5\% jährlicher Deflation in das System zahlen, so zahlt die erste Generation am Anfang $1 Unit$  und am Ende $0.046 Units$. Die nächste Generation beinhaltet keine Deflation, damit das Ergebnis weniger verfälscht wird und einfacher zu Interpretieren ist. 

\paragraph{Ergebnis:}

\begin{table}[hbt!]
\centering
\resizebox{\columnwidth}{!}{%
 \begin{tabular}{|c|c c c|c c c|c c c|} 
 
\hline
\multicolumn{1}{|c}{{}} &
\multicolumn{3}{|c|}{Sim15}\\ [0.5ex] 
\hline
User & Periods & Contributions & Result \\ [0.5ex] 
\hline
1..100 & 1..480 & $sum(1.0 \cdot 1.05^{n}) = 215$ & 189 \\ 
100..200 & 480..960 & $sum(1.0 \cdot 1.05^{40}) = 68$ & 94 \\ [1ex] 
 \hline
 \end{tabular}%
}
\end{table}

Bei diesem Ergebnis sehen wir, dass bei einer Deflation die Units einen höheren Wert und Kaufkraft besitzen und beim Bestehen der Deflationsnevouse erhält der Nutzer aus dem System zur Kaufkraft passende Anzahl an Units.


\subsection{Simulation 3: Langzeit}

Langzeit Simulation versucht über mehrere Generationen das Verhalten des Systems wiederzugeben. Dies soll aufzeigen wie sich das System über mehrere Generationen, beim Anstieg verhält und Abstieg der Nutzerzahlen verhält.

\paragraph{Sim20.} Wir simulieren, dass jedes Jahr 10 Nutzer dazu kommt und Konstanz $1 Unit$ Zahlen, nach 40 Jahren geht jede Generation in die Rente und wir simulieren 190 Jahre und 1120 Nutzern. 

\paragraph{Ergebnis:}

\begin{table}[hbt!]
\centering
\resizebox{\columnwidth}{!}{%
 \begin{tabular}{|c|c c c|c c c|c c c|}  
\hline
\multicolumn{1}{|c}{{}} &
\multicolumn{3}{|c|}{Sim20}\\ [0.5ex] 
\hline
User & Periods & Contributions & Result \\ [0.5ex] 
\hline
1..10 & 1..480 & $1.0 \cdot 480 = 480$ & 952 \\ 
10..20 & 12..492 & $1.0 \cdot 480 = 480$ & 937 \\
$\vdots$ & $\vdots$ & $\vdots$ & $\vdots$ \\
420..430 & 12..984 & $1.0 \cdot 480 = 480$ & 484 \\
430..440 & 12..996 & $1.0 \cdot 480 = 480$ & 477 \\
$\vdots$ & $\vdots$ & $\vdots$ & $\vdots$ \\
1100..1110 & 1320..1800 & $1.0 \cdot 480 = 480$ & 233 \\
1110..1120 & 1332..1812 & $1.0 \cdot 480 = 480$ & 2585 \\ [1ex] 
 \hline
 \end{tabular}%
}
\end{table}

Bei diesem Ergebnis ist es gut zu sehen, dass wenn sich die Anzahl der Zahler kleiner ist als Rentner, in dem Moment fangen die Rentner an weniger aus dem System zu bekommen. Die vorletzten Generationen, verlieren in diesem System. In der letzten Generation die wir als Nachzügler bezeichnen, ist wie geplant eine Belohnung ausgezahlt worden. Hier gibt es Verbesserungspotenzial für die vorletzten Generationen um das System noch fairer zu definieren.

