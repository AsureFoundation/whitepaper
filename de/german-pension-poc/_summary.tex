\section{Fazit}
Im Rahmen dieses Projektes wurde ein Prototyp zum deutschen Rentensystem auf der öffentlichen Ethereum Blockchain entwickelt. Der Prototyp ist im Funktionsumfang stark reduziert und implementiert die Berechnung der Entgeldpunkte (Faktor EP der Rentenformel) als Smart Contract System. Der Zugriff auf das Smart Contract System erfolgt durch eine mobile Webanwendung. Durch diese können Rentenbeiträge mittels der Kryptowährung ETH gezahlt, Renten ausgezahlt und allgemeine Parameter wie z.B. der Zeitpunkt des Renteneintritts spezifiziert werden.

Diese Machbarkeitsstudie des Umlageverfahrens der deutschen Rente auf Ethereum Blockchain hat aufgezeigt, dass der Durchsatz von 7-15 Transaktionen pro Sekunde in Kombination mit anderen Anwendungsfällen für ein Rentensystem in Deutschland nicht ausreicht. Bei Erweiterungen und vollständiger Umsetzung aller Anforderungen, wäre das öffentliche Ethereum Netzwerk heute nicht in der Lage die Verarbeitung ohne großen Aufwand zu unterstützen.

Die Blockchain-Technologie ist Stand 2019 noch in diversen Bereichen limitiert. Es gilt, die Stärken der Blockchain im Kontext von Sozialversicherungen sinnvoll zu nutzen, sodass diese die Schwächen überwiegen. Grundlegende Probleme wie die Skalierung oder der richtige Umgang mit personenbezogenen Daten werden durch diverse Projekte und Institutionen erarbeitet und wir sind optimistisch, dass entsprechende Lösungen in naher Zukunft zur Verfügung stehen.



