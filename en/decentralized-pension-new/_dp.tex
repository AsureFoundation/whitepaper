\section{Dezentrale Rente}
Die dezentrale Rente basiert im Kern auf dem Umlageverfahren und dem
Leistungsprinzip. Je länger ein Beitragszahler in das Rentensystem einzahlt, 
desto länger werden Leistungen im Alter ausgezahlt und je höher die Beiträge,
desto höher die Rentenzahlungen im Alter.

\paragraph{Generische Währungseinheit} Das Design der Rente ist nicht auf eine bestimmte 
Währungseinheit wie z.B. ETH oder BTC festgelegt. Stattdessen verwendet das Design des
Rentensystems die generische Währungseinheit "\textbf{unit}". Diese muss bei der konkreten
Implementierung durch ein konkrete Währung ersetzt wird.

\paragraph{Perioden} Das Rentensytem ist in Perioden unterteilt, welche definiert werden als $P$. Jede Aktion innerhalb des Rentensystems erfolgt
in einer Periode $P[p]$. Die Dauer einer Periode ($P_{duration}$) ist frei wählbar. Innerhalb der folgenden Betrachtung gilt $P_{duration} = 1 month$.

\paragraph{Benutzer} Alle Benutzer des Rentensystems werden als $U$ und individuelle Benutzer als $U[u]$ definiert. Jeder Nutzer hat einen Zustand $U[u]_{state}$ welcher die Zustände \textbf{UC} (Contributor),
\textbf{UP} (Pensioner) und \textbf{UD} (Done) annehmen kann.
Der den Anfangszustand aller Nutzer ist $U[u]_{state} = UC$ 

\subsection{Beitragszahlung}

Beitragszahler können pro Periode $P[p]$ Beiträge zahlen. Der Gesamtbetrag pro Periode und Nutzer wird als $U[u]_{units[p]}$ definiert. Die Gesamtanzahl an Perioden, in denen ein Beitragszahler gezahlt hat, wird als $U[u]_{contrib}$ definiert.

\subsubsection*{Rentenanspruchsperioden}
Die Rentenanspruchsperioden ($U[u]_{pensionperiods}$) eines Rentners legt die Anzahl der Perioden fest,
in denen eine Rentenzahlung erfolgt. Je höher die Anzahl der Beitragsperioden
$U[u]_{contrib}$, desto höher ist die Anzahl der Rentenanspruchsperioden.

Um eine hohe Anzahl an Beitragsperioden zu icentivieren, wird ein Zielwert ($P_{target}$)
für die Anzahl der Beitragsperioden definiert. 

\begin{equation}
	P_{target} = 40 years \cdot 12 months
\end{equation}

Wenn die Anzahl an Beitragsperioden dem Zielwert entspricht, soll die Anzahl der 
Rentenanspruchsperioden genau dem Zielwert entsprechen. Liegt die  Anzahl an
Beitragsperioden unter oder über dem Zielwert, soll die Anzahl der 
Rentenanspruchsperioden entsprechend überproportional kleiner oder größer ausfallen.  

Die Anzahl der Rentenanspruchsperioden ist wie folgt definiert:

\begin{equation}
U[u]_{pensionperiods} = \frac{U[u]_{contrib}^2}{P_{target}}
\end{equation}


\subsubsection*{Dezentrale Rentenpunkte}
Für die gezahlten Beiträge einer Periode ($U[u]_{units[p]}$) erhält ein Beitragszahler Dezentrale Rentenpunkte
($DPT$). Aus der Gesamtanzahl der DPT eines Beitragszahlers wird im Rentenalter
die Höhe der zu zahlenden Rente errechnet.

\begin{equation}
U[u]_{dpt[p]} = DPT(u, p)
\end{equation}

\begin{equation}
DPT(u, p) = DPT_{base}(u, p) \cdot DPT_{bonus}(p)
\end{equation}

\paragraph*{Aktueller Beitragswert}
Der aktuelle Beitragswert (CCV) wird zu Beginn jeder Periode 
$P[n]$ berechnet und als $P[p]_{ccv}$ definiert. Der CCV bildet den Referenzwert für einen $DPT$ der
entsprechenden Periode.

Zur Berechnung des CCV der Periode $P[p]$ wird der CCV der vorherigen
Periode $P[n-1]$ zugrunde gelegt. Ist die Differenz zwischen CCV und dem
durchschnittlichen Beitrag der Periode $P[n-1]$ größer als 10\%, so wird
der CCV der der Periode $P[n]$ entsprechend um 10\% erhöht oder gesenkt.\\
Bei starken Schwankungen des durchschnittlichen Beitrags, nährt sich der CCV langsam an den neuen Durchschnitt an und große Sprünge werden vermieden.


\begin{equation}
P[p]_{units} = avg(\sum_{u=0}^{|U|} U[u]_{units[p]})
\end{equation}

\begin{equation}
P[p]_{ccv} = \begin{cases} 
P[p]_{ccv} \cdot 1.1 & _{if P[p]_{units} \cdot 1.1 > P[p-1]_{ccv}} \\
P[p]_{ccv} \cdot 0.9 & _{if P[p]_{units} \cdot 0.9 < P[p-1]_{ccv}} \\
P[p]_{ccv} & _{otherwise}
\end{cases}
\end{equation}

\paragraph*{Dezentrale Rentenpunkte Basis}
Die Rentenpunkte ist die ERC20 Tokensierung der Kaufkraft in Blockchain-Systemen und ist eine Abstraktion zur Kaufkraft wo der $CCV$-Wert die Referenz abbildet wie die Bereitschaft zur Zahlung in den Vergangenen Perioden vorhanden war. Bei einer Einzahlung die gleich dem CCV-Wert entspricht werden 1.0 DPT Punkte dem Sender gutgeschrieben. Alles über dem CCV Wert werden Maximal 2.0 DPT gutgeschrieben werden, wobei es möglich ist mehr als das doppelte dem CCV-Wert zu bezahlen. Bei einem kleineren Beitrag als dem CCV-Wert wird dem entsprechend Proportional zum min weniger DPT gutgeschrieben werden.

\begin{equation}
DPT_{base}(u, p) = \begin{cases} 
min(\frac{U[u]_{units[p]}} {P[p]_{ccv}}, 2) 
  & _{if U[u]_{units[p]} > P[p]_{ccv}} \\
\frac{U[u]_{units[p]} - min} {P[p]_{ccv} - min} 
  & _{if U[u]_{units[p]} < P[p]_{ccv}} \\
1.0 & _{otherwise}
\end{cases}
\end{equation}

\paragraph*{Dezentrale Rentenpunkte Bonus}
Bis zur Periode $P_{bonus} = 480$ werden zusätzliche Bonus DPT 
($DPT_{bonus}$) an
Beitragszahler ausgegeben. Bonus DPT sollen einen Anreiz für
die ersten Nutzer des Rentensystems schaffen und so die Nutzer belohnen, 
welche früh an das System geglaubt und investiert haben. Das Konzept der Bonus DPT
ist an die aus dem Bitcoin Protokoll bekannte Reduzierung der Mining Rewards angelehnt.

\begin{equation}
DPT_{bonus}(p) = \begin{cases} 
1 + \frac{(P_{bonus} - P[p] + 1)^2}
      {2 \cdot P_{bonus}^2} & _{if P[p] < P_{bonus}} \\
1 & _{otherwise} 
\end{cases}
\end{equation}


\subsubsection{Incentivierung}

Da das System nicht von einer Firma oder Regierung zur Verfügung gestellt wird sondern von der Community dezentralisiert betrieben wird ist es wichtig, dass es Anreize das System so fair wie möglich funktionieren.

Uns war bewusst, dass es nicht möglich sein wird ein System zu Schaffen in dem es nur Gewinner gibt.

Unserer wichtigsten Ziele.

- DAO
- Fair



\begin{itemize}
\item Beitragsjahre => Bezugsjahre
\item Beitragshöhe => Beitragsauszahlung
\item Early Adopter Bonus
\item Bonus für letzte Teilnehmer im System
\end{itemize}


\newpage

\subsection{Pension pre calculation}

\subsubsection*{Settings}

We need some fixed and initial settings,
one of them is 

\begin{equation*}
	min = 0.0000000000000001
\end{equation*}

\begin{equation*}
	years = 40
\end{equation*}

\subsubsection*{entitlementMonths}

\begin{equation*}
	entitlementMonths = \frac{payedMonths^2}{12 \cdot years}
\end{equation*}

\subsubsection*{CCV - Current contribution value}

\begin{equation*}
CCV = \begin{cases} 
CCV * 110\% & _{(unit+10\% > CCV)} \\
CCV * 90\% & _{(unit-10\% < CCV)} \\
CCV & _{(other)}
\end{cases}
\end{equation*}

\subsubsection*{DPT}

\begin{equation*}
DPT = DPT_{base} \cdot DPT_{bonus}
\end{equation*}

\begin{equation*}
DPT_{base} = \begin{cases} 
1 + \frac{unit-max(units)} {CCV - max(units)} 
  & _{(unit > CCV)} \\
\frac{unit - min} {CCV - min} 
  & _{(unit < CCV)} \\
1.0 & _{(other)}
\end{cases}
\end{equation*}

\begin{equation*}
DPT_{bonus} = \begin{cases} 
1.0 + (\frac{(years - year + 1.0)^2}
      {years^2} \cdot 0.5) & _{(year < years)} \\
1.0 & _{(other)} 
\end{cases}
\end{equation*}


\newpage

\subsection{Pension post calculation}

Baisic pension calculation is your points
as factor for contribution pension rate and 
your points as factor for savings pension rate.

\begin{equation*}
pension = DPT \cdot CPR + DPT \cdot SPR + DPT \cdot LPR
\end{equation*}

Die Rente baut auf drei verschiedene Töpfe auf, die Monatlichen Beiträge,
die Gespeicherten Beiträge und die Nachzüglicher Beiträge.

Alle diese Töpfe werden unterschiedlich befüllt.
Die Monatlichen Beiträge werden nur als Durchschnitts Beitrag an die Berechtigten  per Umlageverfahren umverteilt und der Rest wird im Savings hinterlegt.
Bei den Nachzüglern ist die Idee, dass die letzte Generation die das System verlassen würde einen Bonus erhalten würde, dieser Topf wird befüllt,
solange kein Monatsanspruch bei dem Benutzer besteht.

\subsubsection*{Pension from contributions}

\begin{equation*}
	total\_units\_month = 
	\sum_{n=1}^{count} units
\end{equation*}

\begin{equation*}
	total\_weighted\_dpt = 
	\sum_{n=1}^{count} pension\_user\_dpt
\end{equation*}

\begin{equation*}
	weighted\_dpt\_units\_rate = 	
	\frac{total\_units\_month  \cdot avg(units)} 
	{total\_weighted\_dpt}		
\end{equation*}

If $weighted\_dpt\_units\_rate > avg(units)$ then $weighted\_dpt\_units\_rate = avg(units)$
        

\begin{equation*}
	pension = 	
	weighted\_dpt\_units\_rate \cdot user\_dpt
\end{equation*}


\subsubsection*{Pension from saving}

Wenn $weighted\_dpt\_units\_rate < avg(units)$ dann 

\begin{equation*}
	active\_user\_dpts = 
	\sum_{n=1}^{count} active\_user\_dpt
\end{equation*}

\begin{equation*}
	total\_units = 	
	\sum_{p=1}^{periods} \sum_{n=1}^{users} user\_units	
\end{equation*}

\begin{equation*}
	savings\_dpt\_units\_rate = 	
	\frac{total\_units} 
	{active\_user\_dpts \cdot years \cdot 12}		
\end{equation*}

\begin{equation*}
	pension = savings\_dpt\_units\_rate \cdot user\_dpt
\end{equation*}


\subsubsection*{Pension from laggards funds}

Wenn $entitlementMonths >= 1$ dann 

\begin{equation*}
	active\_user\_dpts = 
	\sum_{n=1}^{count} active\_user\_dpt
\end{equation*}

\begin{equation*}
	total\_units = 	
	\sum_{p=1}^{periods} \sum_{n=1}^{users} user\_units	
\end{equation*}

\begin{equation*}
	laggards\_dpt\_units\_rate = 	
	\frac{total\_units} 
	{active\_user\_dpts \cdot years \cdot 12}		
\end{equation*}

\begin{equation*}
	pension = laggards\_dpt\_units\_rate \cdot user\_dpt
\end{equation*}


Voraussetzung ist gegeben wenn $pensioners / periods_open$ genau 1 entspricht