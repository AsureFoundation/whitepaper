\section{Einführung}
Bitcoin ist das weltweit führende digitale Zahlungsmittel auf Basis eines dezentral organisierten Buchungssystems. Die von Bitcoin verwendete Blockchain-Technologie bietet Vorteile im Bereich der Dezentralisierung, der Manipulationssicherheit und der Ausfallsicherheit. Diverse Branchen sind dabei und prüfen, wie sich IT-Systeme und Prozesse, durch die Verwendung der Blockchain-Technologie, optimieren lassen.

\paragraph*{}
Die Asure Stiftung forscht, wie mit Hilfe der Blockchain-Technologie Sozialversicherungssysteme verbessert werden können. Hierzu wurde in Kooperation mit der ECHTNICE GmbH ein Prototyp des deutschen Rentensystems entwickelt. Die Ergebnisse dieser Arbeit werden im folgenden dargestellt.

\subsection{Ziele}

Ziel der Entwicklung des Prototypen zum deutschen Rentensystem ist es, die Blockchain-Technologie als Infrastrukturkomponete zu evaluieren und aufzuzeigen, wie sich diese z.B. von einer IBM Mainframe oder Java-Enterprise basierten Infrastruktur unterscheidet. Des weiteren sollen gewonnene Erkenntnisse in der Umsetzung von komplexen Softwaresystemen als Smart Contract System dokumentiert werden. Hierzu gehören u.a.:

\begin{compactenum}
\item Das Einspielen von Smart Contract Änderungen (z.B. Fehlerkorekturen, funktionale Erweiterungen).
\item Zugriff auf Drittsysteme aus dem Smart Contract System heraus.
\item Verwendung von Entwicklungswerkzeugen und Bibliotheken.
\item Umgang mit verschiedenen Infrastruktumgebungen für z.B. die lokale Entwicklung, Test und Produktion.
\item Entwicklung von modernen Oberflächen, welche Smart Contract Systeme verwenden.
\end{compactenum}

\paragraph*{}
Für die Entwicklung des Prototypen wird die Ethereum Blockchain verwendet. Ethereum ist eine der bekanntesten Blockchain-Projekte, welche die Ausführung von Smart Contracts unterstützt.

\subsection{Abgrenzung}
Im folgenden betrachten wir Anforderungen von Rentensystemen, die im Rahmen dieses Projektes nicht weiter analysiert werden.

\subsubsection{GDPR und der Datenschutz}
Anwendungen, welche auf einer Blockchain betrieben werden und personenbezogene Daten verarbeiten, sind aus den folgenden Gründen problematisch:

\begin{quote}
\textbf{Die Anonymisierung personenbezogener Daten.} Es gibt intensive Diskussionen und derzeit keinen Konsens darüber, was es braucht, um personenbezogene Daten zu anonymisieren, bis zu dem Punkt, an dem die resultierenden Ergebnisse möglicherweise in einem Blockchain-Netzwerk gespeichert werden können. Um ein Beispiel zu nennen: Das Hashing von Daten kann in vielen Situationen nicht als Anonymisierungstechnik betrachtet werden, aber es gibt Fälle, in denen die Verwendung von Hashing zur Erzeugung einzigartiger digitaler Signaturen von Daten, die außerhalb der Kette gespeichert sind, auf einer Blockkette denkbar ist.

\textbf{Die Ausübung einiger Rechte der betroffenen Personen.} Wir weisen darauf hin, dass es bei der Erfassung personenbezogener Daten in einem Blockchain-Netzwerk schwierig sein kann, diese zu korrigieren oder zu entfernen. Die Definition, was als Löschung im Rahmen von Blockketten angesehen werden kann, wird derzeit diskutiert.\cite{gdpr}
\end{quote}

Die GDPR konforme Implementierung des deutschen Rentensystems auf der öffentlichen Ethereum-Blockchain ist explizit kein Ziel dieser Arbeit.

\paragraph*{}
Ob und wie die GDPR konforme Implementierungen des deutschen Rentensystems auf der öffentlichen Ethereum-Blockchain implementiert werden können, soll in einem zukünftigen Projekt erforscht werden. Vielversprechend scheint die Verwendung von sogenannten Zero-Knowledge-Beweisen\footnote{\url{https://de.wikipedia.org/wiki/Zero-Knowledge-Beweis}} und die Überarbeitung bestehender Regelungen in Hinblick auf die Datensparsamkeit.

\subsubsection{Blockchain-Skalierung}
Die öffentliche Ethereum Blockchain kann 7-15 Transaktionen pro Sekunde (TPS) verarbeiten \cite{vitalikscale}. Pro Monat stehen demnach ca. 18-39 Millionen Transaktionen zur Verfügung. Auf der öffentlichen Ethereum Blockchain laufen diverse Anwendungen verschiedener Hersteller die sich dementsprechend auch gegenseitig beeinflußen. Benötigt eine Anwendung besonders viele Transaktionen, stehen diese dementsprechend nicht mehr für andere Anwendungen zur Verfügung \cite{cryptokitty}.  

2018 umfasste die deutsche Rentenversicherung ca. 37,5 Millionen aktiv Versicherte und ca. 21,04 Millionen Rentner. \cite{rente2018,rentezahlen2019} Für ein Blockchain-basiertes Rentensystem, welches für jede Ein-, und Auszahlung genau eine Transaktion benötigt, würden so ca. 58,04 Millionen Transaktionen pro Monat erforderlich sein.

Dies ist nur eine simple Betrachtung und je nach Implementierung kann die Anzahl der benötigten Transaktionen für ein Blockchain-basiertes Rentensystem für Deutschland stark reduziert werden. Dennoch wird deutlich, dass insbesondere die öffentliche Ethereum Blockchain nicht über die entsprechenden Kapazitäten verfügt.

\paragraph*{}
Das Problem der Blockchain-Skalierung ist ein bekanntes Problem zu welchem diverse Lösungsvorschläge erforscht und erarbeitet werden \cite{scaling}. Im Rahmen dieser Arbeit wird das Thema Blockchain-Skalierung nicht weiter betrachtet.

\subsubsection{Alternative Blockchain-Projekte}
Im Rahmen dieses Projektes haben wir uns ausschließlich auf die öffentliche Ethereum Blockchain und die dazugehörige Programmiersprache Solidity fokusiert.

\paragraph*{}
Neben der öffentlichen Ethereum Blockchain, kann Ethereum auch in privaten oder konsortium basierten Szenarien verwendet werden. Der Einsatz in privaten oder konsortium basierten Szenarien kann sich u.U. stark von dem öffentlichen Szenario hinsichtlich Skalierung, Performance und dem Umgang mit personenbezogenen Daten unterscheiden und wurde explizit nicht betrachtet.

\paragraph*{}
Auch alternative Blockchain-Projekte wie z.B. 
Hyperledger Fabric\footnote{\url{https://www.hyperledger.org/projects/fabric}}, 
Qtum\footnote{\url{https://qtum.org/en}}
oder EOS\footnote{\url{https://eos.io/}}
werden im Rahmen dieser Arbeit nicht betrachtet. 