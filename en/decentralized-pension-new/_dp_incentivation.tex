\subsection{Pension post calculation}

Baisic pension calculation is your points
as factor for contribution pension rate and 
your points as factor for savings pension rate.

\begin{equation*}
pension = DPT \cdot (CPR + SPR + LPR)
\end{equation*}

Die Rente baut auf drei verschiedene Töpfe auf, die Monatlichen Beiträge,
die Gespeicherten Beiträge und die Nachzüglicher Beiträge.

Alle diese Töpfe werden unterschiedlich befüllt.
Die Monatlichen Beiträge werden nur als Durchschnitts Beitrag an die Berechtigten  per Umlageverfahren umverteilt und der Rest wird im Savings hinterlegt.
Bei den Nachzüglern ist die Idee, dass die letzte Generation die das System verlassen würde einen Bonus erhalten würde, dieser Topf wird befüllt,
solange kein Monatsanspruch bei dem Benutzer besteht.

\subsubsection*{Pension from contributions}

\begin{equation*}
	total\_units\_month = 
	\sum_{n=1}^{count} units
\end{equation*}

\begin{equation*}
	total\_weighted\_dpt = 
	\sum_{n=1}^{count} pension\_user\_dpt
\end{equation*}

\begin{equation*}
	weighted\_dpt\_units\_rate = 	
	\frac{total\_units\_month  \cdot avg(units)} 
	{total\_weighted\_dpt}		
\end{equation*}

If $weighted\_dpt\_units\_rate > avg(units)$ then $weighted\_dpt\_units\_rate = avg(units)$
        

\begin{equation*}
	pension = 	
	weighted\_dpt\_units\_rate \cdot user\_dpt
\end{equation*}


\subsubsection*{Pension from saving}

Wenn $weighted\_dpt\_units\_rate < avg(units)$ dann 

\begin{equation*}
	active\_user\_dpts = 
	\sum_{n=1}^{count} active\_user\_dpt
\end{equation*}

\begin{equation*}
	total\_units = 	
	\sum_{p=1}^{periods} \sum_{n=1}^{users} user\_units	
\end{equation*}

\begin{equation*}
	savings\_dpt\_units\_rate = 	
	\frac{total\_units} 
	{active\_user\_dpts \cdot years \cdot 12}		
\end{equation*}

\begin{equation*}
	pension = savings\_dpt\_units\_rate \cdot user\_dpt
\end{equation*}


\subsubsection*{Pension from laggards funds}

Wenn $entitlementMonths >= 1$ dann 

\begin{equation*}
	active\_user\_dpts = 
	\sum_{n=1}^{count} active\_user\_dpt
\end{equation*}

\begin{equation*}
	total\_units = 	
	\sum_{p=1}^{periods} \sum_{n=1}^{users} user\_units	
\end{equation*}

\begin{equation*}
	laggards\_rate = 	
	\frac{total\_units} 
	{active\_user\_dpts \cdot years \cdot 12}		
\end{equation*}

\begin{equation*}
	pension = laggards\_rate \cdot user\_dpt
\end{equation*}


Voraussetzung ist gegeben wenn $pensioners / periods_open$ genau 1 entspricht