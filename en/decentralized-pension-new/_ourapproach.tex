\subsection{Our Approach: Decentralized Pension}

Wir sind überzeugt, dass es auch möglich ist Sozialversicherungen in einer Dezentralen Manier zu designen ohne Einfluss von Firmen oder Regierungsbehörden möglich ist.

Wir orientieren uns an den gängigen Umlageverfahren. Für das Umlageverfahren sind verschiedene Voraussetzungen relevant. 

\begin{itemize}
\item Betragspflicht - sogt dafür, dass sicher gestellt wird, dass die Beiträge ralative Konstanz haben
\item Beitragswert der meistens Lohnorientiert ist - dies ermöglicht eine gewisse sicherheit und representiert in gewisser weise die Kaufkraft
\end{itemize}

Da wir aus geopolitischen gründen nicht möglich auf diese Voraussetzungen wert zu legen, deswegen haben wir alternativen erarbeitet.
Wir schaffen Anreize um bei der Beitragspflicht nachhelfen zu können und bei dem Beitragswert führen wir einen von Mitgliedern representativen Referenzbeitragssatz der von den Mitgliedern dynamisch sich an das Verhalten anpasst.

Die Vorteile dieses Systems sind:
\begin{table}[]
\begin{tabular}{p{5cm}|p{8cm}}
Decentralized & Mit hilfe der Blockchain technologie ist das System dezantralisiert verfügbar.  \\ \hline
Autonomous & Es sind keine Mitarbeiter Notwendig um das system betreiben zu müssen   \\ \hline
Permissionless & Der Zugang zu diesem System ist für jeden gleich zugänglich. \\ \hline
Reduce costs & Durch die Automatisierung und Dezentralisierten Betrieb gibt es keinen weiteren Kostenaufwand ausser den Transaktionsgebühren. \\ \hline
Without any intermediary & Dieses System soll der Community zur Verfügung gestellt werden um Aufzuzeigen, dass es auch anders funktionieren kann und die Armut auf der Welt kann dadurch reduziert werden. \\ \hline
Corruption-proof & Es gibt keine Möglichkeit das Geld entwenden zu können. \\ \hline
Tamper-proof & Die erlaubten veränderungen des Systems sind den Mitgliedern überlassen. \\ \hline
Froud-proof & Betrug wird dadurch vermieden, dass wir keine Externen Informationen für den betrieb benötigen. \\ \hline
Haritable & Der Anspruch an eine Rente kann durch die Weitergabe des Privaten-Schlüssels übergeben werden.
\end{tabular}
\end{table}



Dezentrale Rente mittels Umlageverfahren und Incentivierung als SmartContract.
Global Verfügbar. Keine Verwaltungskosten (außer Tx Kosten) oder Provision

\subsubsection{Umlageverfahren}

\begin{itemize}
\item Beitragszahlung und Berechnung DPT
\item Rentenzahlung und Umverteilung der monatichen Beiträge
\item Rentenzahlung und Umverteilung der Rücklagen
\end{itemize}

\subsubsection{Incentivierung}

\begin{itemize}
\item Beitragsjahre = Bezugsjahre
\item Early Adopter Bonus
\item Bonus für letzte Teilnehmer im System
\end{itemize}