\section{Fazit}
Die Technologie ist heute in mehreren Bereichen limitiert und genau in diesen Bereichen wird auch viel geforscht.

Diese Machbarkeitsstudie des Umlageverfahrens der deutschen Rente auf Ethereum Blockchain hat aufgezeigt, dass der Durchsatz der Transaktionen 7-15Tx/s in Kombination mit anderen Anwendungsfällen für ein Rentensystem in Deutschland nicht ausreicht. 
Die Sonderfälle der deutschen Rente wurden in dieser Arbeit nicht Berücksichtigt, bei Erweiterungen und vollständiger Umsetzung aller Regeln uns statistischen Daten, wäre das öffentliche Ethereum Netzwerk heute nicht in der Lage die Verarbeitung ohne großen Aufwand zu unterstützen.
Die Massendatenverarbeitung die durch Oracles, IPFS heute realisiert werden, sind noch nicht ausgereift für einen professionellen Einsatz und benötigen weitere alternativen.
Datenschutz und Privatsphäre sind ein wichtiger Aspekt der zwischen Transparenz und Schutz der Daten und Privatsphäre in Balance gefunden werden muss. 
Obwohl es Bereiche gibt, die noch Herausforderungen stellen, zeigt es, dass wenn diese Herausforderungen gelöst werden, ist der Betrieb eines Umlagebasierten Rentensystems auf einer öffentlichen Blockchain ein sehr realistisches Szenario für die Zukunft. Damit es möglich wird, ist es wichtig diese Themen so früh wie möglich zu fördern und zu unterstützen.

