\section{Dezentrale Rente}

Die dezentrale Rente basiert im Kern auf dem Umlageverfahren und dem
Leistungsprinzip. Je länger ein Beitragszahler in das Rentensystem einzahlt, 
desto länger werden Leistungen im Alter ausgezahlt und je höher die Beiträge,
desto höher die Rentenzahlungen im Alter.

\subsection{Beitragszahlung}

\subsubsection*{Rentenanspruchsperioden}

Die Rentenanspruchsperioden ($P_{pension}$) legt die Anzahl der Perioden fest,
in denen eine Rentenzahlung erfolgt. Je höher die Anzahl der Beitragsperioden
($P_{contrib}$), desto höher ist die Anzahl der Rentenanspruchsperioden.

Um eine hohe Anzahl an Beitragsperioden zu icentivieren, wird ein Zielwert 
für die Anzahl der Beitragsperioden ($P_{contribTarget}$) definiert. 

\begin{equation}
	P_{contribTarget} = 40 years \cdot 12 months
\end{equation}

Wenn die Anzahl an Beitragsperioden dem Zielwert entspricht, soll die Anzahl der 
Rentenanspruchsperioden genau dem Zielwert entsprechen. Liegt die  Anzahl an
Beitragsperioden unter oder über dem Zielwert, soll die Anzahl der 
Rentenanspruchsperioden entsprechend überproportional kleiner oder größer ausfallen.  

Die Anzahl der Rentenanspruchsperioden ist wie folgt definiert:

\begin{equation}
	P_{pension} = \frac{P_{contrib}^2}{P_{contribTarget}}
\end{equation}


\subsubsection*{Dezentrale Rentenpunkte}

Für die gezahlten Beiträge erhält ein Beitragszahler Dezentrale Rentenpunkte
($DPT$). Aus der Gesamtanzahl der DPT eines Beitragszahlers wird im Rentenalter
die Höhe der zu zahlenden Rente errechnet.

\begin{equation}
DPT = DPT_{base} \cdot DPT_{bonus}
\end{equation}

\paragraph*{Aktueller Beitragswert}

Der aktuelle Beitragswert ($CCV$) wird zu Beginn jeder Periode 
$P_{contrib}{n}$ berechnet und bildet den Referenzwert für einen $DPT$ der
entsprechenden Periode.

Zur Berechnung des $CCV$ der Periode $P_{contrib}{n}$ wird der $CCV$ der vorherigen
Periode $P_{contrib}{n-1}$ zugrunde gelegt. Ist die Differenz zwischen $CCV$ und dem
durchschnittlichen Beitrag der Periode $P_{contrib}{n-1}$ größer als 10\%, so wird
der $CCV$ der der Periode $P_{contrib}{n}$ entsprechend um 10\% erhöht oder gesenkt.
Bei starken Schwankungen des durchschnittlichen Beitrags, nährt sich der $CCV$ langsam an den neuen Durchschnitt an und große Sprünge werden vermieden.

\begin{equation}
CCV = \begin{cases} 
CCV * 110\% & _{(unit+10\% > CCV)} \\
CCV * 90\% & _{(unit-10\% < CCV)} \\
CCV & _{(other)}
\end{cases}
\end{equation}

\paragraph*{Dezentrale Rentenpunkte Basis}

\begin{equation}
DPT_{base} = \begin{cases} 
min(\frac{unit} {CCV}, 2) 
  & _{(unit > CCV)} \\
\frac{unit - min} {CCV - min} 
  & _{(unit < CCV)} \\
1.0 & _{(unit = CCV)}
\end{cases}
\end{equation}

\paragraph*{Dezentrale Rentenpunkte Bonus}

Bis zur Periode $P_{bonus} = 480$ werden zusätzliche Bonus DPT 
($DPT_{bonus}$) an
Beitragszahler ausgegeben. Diese Bonuspunkte sollen einen Anreiz für
die ersten Nutzer schaffen und so die belohnen, welche früh an das System geglaubt und investiert haben.

\begin{equation}
DPT_{bonus} = \begin{cases} 
1 + (\frac{(P_{bonus} - P_{n} + 1)^2}
      {P_{bonus}^2} \cdot .5) & _{(P_{n} < P_{bonus})} \\
1 & _{(other)} 
\end{cases}
\end{equation}


%Das Versprechen der dezentralen Rente ist wie folgt: Beitragszahler, welche jeden Monat %einen Beitrag in Höhe der durchschnittlichen Beitragshöhe pro Periode zahlen, bekommen im %Alter eine Rentenzahlung pro in der Höhe der 



% Beitragszahler zahlen über die gesamte Dauer des Zahlungszeitraum einen Beitrag pro %Zahlungsperiode. Für die gezahlten Beiträge erhält der Beitragszahler "Dezentrale %Rentenpunkte (DPT)". Ein DPT entspricht dem Wert des durchschnittlichen Beitrags einer %Zahlungsperiode. Zahlt ein Beitragszahler mehr als den Durchschnitt, bekommt er 


Beitragszahler zahlen in Regelmäßigen Perioden P einen Beitrag X in das Rentensystem und erhalten 

\subsection{Rentenzahlung}

%\subsection{Umlageverfahren}

\begin{itemize}
\item Beitragszahlung und Berechnung DPT
\item Rentenzahlung und Umverteilung der monatichen Beiträge
\item Rentenzahlung und Umverteilung der Rücklagen
\end{itemize}
\subsection{Rentenzahlung}

Beitragszahler können frei wählen, wann Sie in Rente gehen. Geht ein Beitragszahler in Rente, können keine Beitragszahlungen mehr getätigt werden und je nach gezahlten Beiträgen, können stattdessen Rentenauszahlungen erfolgen. Der Übergang von Beitragszahler zu Renter wird durch die Änderung des Zustandes $U[u]_{state} = UP$ makiert.

Die Berechnung der zu zahlenden Rente erfolgt anhand der Gesamtanzahl der DPT eines Rentners ($U[u]_{dpt\_total}$).

\begin{equation}
U[u]_{dpt\_total} = \sum_{p=0}^{|P|} U[u]_{dpt[p]}
\end{equation}

Die Rentenanspruchsperioden $U[u]_{pensionperiods}$ legen fest, in wie vielen Perioden eine Rente ausgezahlt wird. In
welchen Perioden ein Rentner seine Rentenzahlungen in Anspruch nimmt, bleibt diesem selbst überlassen und kann frei
gewählt werden. Wurden alle Rentenzahlungen in Anspruch genommen, wird der Zustand ensprechend durch $U[u]_{state} = UD$ ersetzt.

Hat ein
Rentner in seinen Beitragsperioden immer den durchschnittlichen Beitrag in das
Rentensystem eingezahlt, soll seine
Rentezahlung auch den durchschnittlichen Beitragszahlungen der aktuellen
Periode entsprechen und so die in $DPT$ gespeicherte Kaufkraft wieder hergestellt werden.


Die Dezentrale Rente verfügt über drei Geldtöpfe (Beiträge, Rücklagen und Bonus) 
aus denen die Rente zusammengesetzt und ausgezahlt wird. Die Beiträge werden wie folgt auf 
die drei Geldtöpfe verteilt:
\begin{compactenum}
\item Beiträge von Beitragszahlern, welche noch über keine Rentenanspruchsperioden ($U[u]_{pensionperiods} = 0$) verfügen, werden in den Bonustopf abgelegt.
\item Beiträge von Beitragszahlern, welche über Rentenanspruchsperioden ($U[u]_{pensionperiods} > 0$) verfügen, werden in dem Beitragstopf abgelegt.
\item Der Restwert des Beitragstopf wird nach Auszahlung der Renten einer Periode in den als Rücklage verwendet.
\end{compactenum}

Für jeden Geldtopf wird eine Umrechnungsrate berechnet, welche pro 
Periode festlegt, wie viel ein $DPT$ aus dem entsprechenden
Geldtopf Wert ist.

Die Rentenzahlung wird somit wie foglt definiert:

\begin{equation}
U[u]_{pension[p]} = U[u]_{dpt\_total} \cdot (CPR(p) + SPR + LPR)
\end{equation}

\subsubsection*{Beitragsrentenrate ($CPR$)}
Alle Beiträge einer Beitragsperiode ($P[p]units$) werden gesammelt und anteilig an Rentner ausgezahlt. Die Beitragsrentenrate $CPR(p)$ definiert die Umrechnungsrate von DPT zu Beiträgen.

\begin{equation}
P[p]_{units} = \sum_{u=0}^{|U|} U[u]_{units[p]}
\end{equation}

Die zu zahlenden Renten sind durch die durchschnittliche Beitragszahlung der 
jeweiligen Periode ($avg(P[p]units)$) gedeckelt. Gibt es mehr Beitragszahler als Rentner im System, werden so eventuelle Überschüsse als Rücklage verwendet und nicht direkt ausgezahlt. Reichen die Beitragszahlungen nicht, um die durchschnittliche Beitragszahlung der 
jeweiligen Periode zu zahlen, werden die Beiträge proportional anhand der DPT auf alle Rentner und Rentenzahlungen  aufgeteilt.

\begin{equation}
P[p]_{dpt\_pensioner} = 
\sum_{u=0}^{|U|} \begin{cases} 
U[u]_{dpt\_total[p]} & _{if U[u]_{state} = UP}\\
0 & _{otherwise}
\end{cases}
\end{equation}

\begin{equation*}
CPR_{a}(p) = \frac{avg(P[p]_{units})}{P_{target}}
\end{equation*}

\begin{equation*}
CPR_{b}(p) = \frac{P[p]_{units}} {P[p]_{dpt\_pensioner} \cdot avg(P[p]_{units}))}
\end{equation*}

\begin{equation}
CPR(p) = min(CPR_{a}(p), CPR_{b}(p))
\end{equation}


\subsubsection*{Rücklagenrentenrate ($SPR$)}

% auc = active user count

\begin{equation}
P[p]_{auc} = \sum_{u=0}^{|U|} \begin{cases} 
0 & _{if U[u]_{state} = DP}\\
1 & _{otherwise}
\end{cases}
\end{equation}

\begin{equation}
P[p]_{active\_dpt} = \sum_{u=0}^{|U|} \begin{cases} 
0 & _{if U[u]_{state} = DP}\\
U[u]_{dpt\_total} & _{otherwise}
\end{cases}
\end{equation}

\begin{equation*}
TOPP(u) = \sum_{p=0}^{|P|} \begin{cases}
1 & _{if U[u]_{pension[p]} > 0}\\
0 & _{otherwise}\\
\end{cases}
\end{equation*}


% top = total open periods

\begin{equation}
P[p]_{top} = \sum_{u=0}^{|U|} \begin{cases} 
P_{target} - TOPP(u)  & _{if U[u]_{state} = DP}\\
P_{target} & _{if U[u]_{state} = DC}\\
0 & _{otherwise}
\end{cases}
\end{equation}

\begin{equation}
SPR(p) = \frac{total\_savings - laggarts\_savings} {P[p]_{active\_dpt} \dot {\frac{P[p]_{top}} {P[p]_{auc}}}
}
\end{equation}


\subsubsection*{Bonusrentenrate ($LPR$)}

Wenn $entitlementMonths >= 1$ dann 

\begin{equation*}
	active\_user\_dpts = 
	\sum_{n=1}^{count} active\_user\_dpt
\end{equation*}

\begin{equation*}
	total\_units = 	
	\sum_{p=1}^{periods} \sum_{n=1}^{users} user\_units	
\end{equation*}

\begin{equation*}
	laggards\_rate = 	
	\frac{total\_units} 
	{active\_user\_dpts \cdot years \cdot 12}		
\end{equation*}

\begin{equation*}
	pension = laggards\_rate \cdot user\_dpt
\end{equation*}


Voraussetzung ist gegeben wenn $pensioners / periods_open$ genau 1 entspricht