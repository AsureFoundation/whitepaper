\section{Simulations}

As part of our research, we performed several simulations of the decentralized pension model. The goal is to simulate different user behavior and to further optimize the incentives and the model and to check its carrying capacity. Another aim of the simulations is to identify user groups who benefit from the pension system, as well as the ones who suffer losses.

All simulations were developed in the programming language $Rust$ and published on Github under the MIT license.\\
Github: \url{https://bit.ly/2T73llg}

\subsection{Simulation 1: Zero-Win}

The first simulations try to determine that there will be no losses in the system, if all users behave equally fair and constant.

\paragraph{Sim01:} 100 users pay $1.0 unit$ into the pension system and retire at the same time.

\paragraph{Sim02:} 90 users pay $1.0 unit$ and 10 users pay $2.0 units$ into the pension system and retire at the same time.

\paragraph{Sim03:} 90 users pay $1.0 unit$ and 10 users pay $0.1 units$ into the pension system and retire at the same time.

\paragraph{Outcome:}

\begin{table}[hbt!]
\centering
\resizebox{\columnwidth}{!}{%
 \begin{tabular}{|c|c c c|} 
 
\hline
\multicolumn{1}{|c}{{}} &
\multicolumn{3}{|c|}{Sim01} \\ [0.5ex] 
\hline
User & Periods & Contributions & Result \\ [0.5ex] 
\hline
1..10 & 1..480 & $1.0 \cdot 480 = 480$ & 480\\ 
20..100 & 1..480 & $1.0 \cdot 480 = 480$ & 480\\ [1ex] 
 \hline
 \end{tabular}%
}
\end{table}

\begin{table}[hbt!]
\centering
\resizebox{\columnwidth}{!}{%
 \begin{tabular}{|c|c c c|} 
 
\hline
\multicolumn{1}{|c}{{}} &
\multicolumn{3}{|c|}{Sim02}\\ [0.5ex] 
\hline
User & Periods & Contributions & Result \\ [0.5ex] 
\hline
1..10 & 1..480 & $2.0 \cdot 480 = 960$ & 960\\ 
20..100 & 1..480 & $1.0 \cdot 480 = 480$ & 480 \\ [1ex] 
 \hline
 \end{tabular}%
}
\end{table}

\begin{table}[hbt!]
\centering
\resizebox{\columnwidth}{!}{%
 \begin{tabular}{|c|c c c|} 
 
\hline
\multicolumn{1}{|c}{{}} &
\multicolumn{3}{|c|}{Sim03} \\ [0.5ex] 
\hline

 User & Periods & Contributions & Result \\ [0.5ex] 
 \hline
1..10 & 1..480 & $0.1 \cdot 480 = 48$ & 48 \\ 
20..100 & 1..480 & $1.0 \cdot 480 = 480$ & 480 \\ [1ex] 
 \hline
 \end{tabular}%
}
\end{table}

The simulation shows that with equal and fair behavior and no influence of inflation and deflation there are no winners and no losers, everyone just gets their deposits back without any influence of inflation or deflation in economics.

\subsection{Simulation 2: Inflation/Deflation}

TODO: revalued vs upvalued?
In dieser Simulationsreihe wird das Verhalten von Inflation und Deflation der Units simuliert. Es ist die Entwertung und Aufwertung der Units, welches einen Einfluss auf den Benutzer hat. Bei einer Abwertung würden die Nutzer mehr Bezahlen und bei einer Aufwertung dementsprechend weniger wobei die Kaufkraft gleich bleiben würde. Durch $CCV$ wird dies automatisch aufgefangen und die Renten Ein- und Auszahlungsverhalten dementsprechend angepasst. 
This simulation series simulates the behavior of inflation and deflation of the units. It is the devaluation and revaluation of the units that has an influence on the user. If the units were devalued, the users would pay more and if the units were revalued, the purchasing power would remain the same. By $CCV$ this is automatically absorbed and the pension payment and payout behavior is adjusted accordingly. 

\paragraph{Sim14:} Todo: fehlt? oder sinkt
Bei einer Inflation von 5\% fehlt der Wert der Units und dementsprechend Zahlen die Benutzer mehr Units. Nehmen wir 2 Generationen mit je 100 Nutzern und lassen diese bei 5\% jährlicher Inflation in das System zahlen, so zahlt die erste Generation am Anfang $1 Unit$  und am Ende $21.725 Units$. Die nächste Generation beinhaltet keine Inflation, damit das Ergebnis weniger verfälscht wird und einfacher zu Interpretieren ist. 
With an inflation of 5\% the value of the units is missing and accordingly the user numbers more units. If we take 2 generations with 100 users each and let them pay into the system at 5\% annual inflation, the first generation pays $1 Unit$ at the beginning and $21,725 Units$ at the end. 

\paragraph{Outcome:}

\begin{table}[hbt!]
\centering
\resizebox{\columnwidth}{!}{%
 \begin{tabular}{|c|c c c|c c c|c c c|} 
 
\hline
\multicolumn{1}{|c}{{}} &
\multicolumn{3}{|c|}{Sim14}\\ [0.5ex] 
\hline
User & Periods & Contributions & Result \\ [0.5ex] 
\hline
1..100 & 1..480 & $\sum_{j=0}^{39} 12 \cdot 1.05^j = 1456$ & 2320 \\ 
100..200 & 480..960 & $\sum_{j=40}^{79} 12 \cdot 1.05^{j} = 3380$ & 2514 \\ [1ex] 
 \hline
 \end{tabular}%
}
\end{table}

With this result we see that in the case of inflation the units are devalued and in the case of existing inflation the user receives from the system the number of units matching the purchasing power.

\paragraph{Sim15:} In a deflation, where the value of units increases, the user will pay correspondingly less into a system. If we take 2 generations with 100 users each and let them pay into the system at 5\% annual deflation, the first generation pays $1 Unit$ at the beginning and $0.046 Units$ at the end. 

\paragraph{Outcome:}

\begin{table}[hbt!]
\centering
\resizebox{\columnwidth}{!}{%
 \begin{tabular}{|c|c c c|c c c|c c c|} 
 
\hline
\multicolumn{1}{|c}{{}} &
\multicolumn{3}{|c|}{Sim15}\\ [0.5ex] 
\hline
User & Periods & Contributions & Result \\ [0.5ex] 
\hline
1..100 & 1..480 & $sum(1.0 \cdot 1.05^{n}) = 215$ & 189 \\ 
100..200 & 480..960 & $sum(1.0 \cdot 1.05^{40}) = 68$ & 94 \\ [1ex] 
 \hline
 \end{tabular}%
}
\end{table}

Todo: Deflationsnevouse
Bei diesem Ergebnis sehen wir, dass bei einer Deflation die Units einen höheren Wert und Kaufkraft besitzen und beim Bestehen der Deflationsnevouse erhält der Nutzer aus dem System zur Kaufkraft passende Anzahl an Units.
With this result, we see that in the case of deflation the units have a higher value and purchasing power, and when the deflations exist, the user receives from the system the purchasing power appropriate number of units.


\subsection{Simulation 3: Long-term}

Long-term simulation tries to reproduce the behavior of the system over several generations. It should show how the system behaves over several generations, during the increase and behold decline in user numbers.

\paragraph{Sim20:} We simulate that every year 10 users come in and constantly pay $1 unit$, after 40 years every generation retires and we simulate 190 years and 1120 users.

\paragraph{Outcome:}

\begin{table}[hbt!]
\centering
\resizebox{\columnwidth}{!}{%
 \begin{tabular}{|c|c c c|c c c|c c c|}  
\hline
\multicolumn{1}{|c}{{}} &
\multicolumn{3}{|c|}{Sim20}\\ [0.5ex] 
\hline
User & Periods & Contributions & Result \\ [0.5ex] 
\hline
1..10 & 1..480 & $1.0 \cdot 480 = 480$ & 952 \\ 
10..20 & 12..492 & $1.0 \cdot 480 = 480$ & 937 \\
$\vdots$ & $\vdots$ & $\vdots$ & $\vdots$ \\
420..430 & 12..984 & $1.0 \cdot 480 = 480$ & 484 \\
430..440 & 12..996 & $1.0 \cdot 480 = 480$ & 477 \\
$\vdots$ & $\vdots$ & $\vdots$ & $\vdots$ \\
1100..1110 & 1320..1800 & $1.0 \cdot 480 = 480$ & 233 \\
1110..1120 & 1332..1812 & $1.0 \cdot 480 = 480$ & 2585 \\ [1ex] 
 \hline
 \end{tabular}%
}
\end{table}

With this result, it is easy to see that if the number of payers is smaller than retirees, at that moment pensioners start to get less out of the system. The penultimate generations lose in this system. In the last generation, which we call latecomers, a reward has been paid as planned. There is room for improvement for the last but one generations to make the system even fairer.

