\section{Conclusion}

This paper proposes the development of a decentralized, globally available pension system using the redistribution scheme and the blockchain technology.

It shows how the redistribution of contributions and pension payouts can be implemented without depending on personal information of the insured person (e.g. age, identity, etc.). 
%Traditionally, redistribution schemes work best if they are mandatory as it ensures steady contributions and therefore pension payouts. 
Also, this paper proposes a redistribution scheme with voluntary participation and proper incentivation to guarantee a steady number of new contributors. Contributors have an incentive to participate in the pension system as they are rewarded with a higher pension if they play by the rules of the pension system. 
Due to its decentralized design, the pension system is not controlled by anybody, is globally available, and open for everyone to participate. The rules of the proposed pension system are immutable and transparently saved onto the blockchain. 

If everyone adheres to the same rules, the only thing that occurs is that the system store the purchasing power. Without any influence of inflation or deflation in economics, the same number of $Units$ paid into the system will be paid out to users. Therefore there is no profit and no loss, but the value retention was simply being kept for years. \footnote{ The only costs incurred will be the transaction costs, in case of a private network, these can be further minimized.}

It is assumed that some users will opt for an earlier pension, even if it means partial loss for a user. Some users will lose full access to the system because e.g. the private key is lost, the pension contributions are not inherited and not collected. After some time, the paid-in values will be released, which will benefit the other users in the system.


