\section{Decentralized social security}
There are different requirements depending on the type of social insurance. This also has an impact on the plasma implementation. This chapter will take a closer look on the different requirements of these systems. 

\subsection{Pension}
A pension system consists of a number of contributors and pensioners. Every contributor pays a premium each month. In some systems, the premiums get paid by the company on behalf of the contributor which would mean a massive reduction of transactions needed.
On the other side, all pensioners get their pension from the pension system. This usually happens at a fixed date and all pensioners get paid at the same time. This makes it an ideal use case for mass payout transactions.

\subsection{Unemployment}
Unemployment insurance is the protection against loss of work. Participants who have a job pay a contribution where in case of loss of work the time is bridged by the contributors to find a job again.

\subsection{Health care}
The parties in healthcare are diverse - there are insured persons who make deposits, there are doctors, hospitals, pharmacies and other service providers who issue invoices, these can be offset against the system or via the insured person who submits the invoices to the system and gets the costs reimbursed. Here there are different possibilities how you can realize the processing in batches, the insured can submit the accumulated invoices at the end of the year or the doctors, hospitals, pharmacies and other service providers can also submit their collective invoices in batches.

\subsection{Other social security schemes}
The variable requirements show that the requirements for implementation are manifold and that there are a wide variety of variants for parties with different interests. As a result, it is a challenge to design the side-chains with mass deposits and mass exits.

