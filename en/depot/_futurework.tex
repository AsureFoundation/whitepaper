\section{Future work}


The present work aimed to develop a concept for the decentralized pension. In closing, it can be noted that there are still open issues to be discussed.


\paragraph{Ethereum Smart-Contract.} The next step is to translate the specification in this paper into Smart Contract on Ethereum network and make it available to the community for review.

\paragraph{Scalable network.} In order for the system to be able to handle multi-product as well as mass transactions support, it is necessary to develop a scalable network, especially for decentralized pension and, in general, decentralized social insurance systems.

\paragraph{Use of stablecoins.} There is still a lack of an evaluation of the ways to use stablecoins  to protect deposits against volatility in a pay-as-you-go process.

\paragraph{Pension payouts resolution.} There is still no research done for the case if certain claims are not retrieved for a number of years.

\paragraph{Reset:} For a full resolution, it would be interesting to reset the system so that the initialization values can be easily reset.

\paragraph{Liability, Governance.} The question of the liability and control of the system is not yet fully clarified.

\paragraph{Asset management.} In addition to a pure pay-as-you-go system, deposits can be made directly into the system, which involves both risks and reward opportunities, and for those who are more interested in risks, such research and results would be of great interest.

\paragraph{Other use cases.} In addition to decentralized pension, other social insurances may also be decentralized, such as: health insurance, accident insurance, unemployment insurance etc. In addition to the well-known social insurance, the future and unconditional basic income can be organized decentrally. 

\paragraph{Release unused pensions} It is assumed that some users will opt for an earlier pension, even if it means partial loss for a user. Some users will lose full access to the system because e.g. the private key is lost, the pension contributions are not inherited and not collected. After some time, the paid-in values will be released, which will benefit the other users in the system.

