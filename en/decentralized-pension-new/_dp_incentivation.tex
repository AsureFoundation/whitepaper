\subsection{Rentenzahlung}
Die Berechnung der zu zahlenden Rente wird anhand der $DPT$ eines Rentners bestimmt.
Hat ein
Rentner in seinen Beitragsperioden immer den durchschnittlichen Beitrag in das
Rentensystem eingezahlt, soll seine
Rente auch dem durchschnittlichen Beitrag der aktuellen
Periode entsprechen und so die in $DPT$ gespeicherte Kaufkraft wieder hergestellt werden.

\begin{equation}
pension = DPT \cdot (CPR + SPR + LPR)
\end{equation}

Die Dezentrale Rente verfügt über drei Geldtöpfe (Beiträge, Rücklagen und Bonus) 
aus denen die Rente zusammengesetzt und 
ausgezahlt wird. Für jeden Geldtopf wird eine Umrechnungsrate berechnet. Die
Umrechnungsraten legen pro Periode fest, wie viel ein $DPT$ aus dem entsprechenden
Geldtopf Wert ist.

\subsubsection*{Beitragsrentenrate ($CPR$)}
Alle Beiträge einer Beitragsperiode werden gesammelt und anteilig an die jeweils
aktiven Rentner ausgezahlt.
Die zu zahlenden Renten sind nach oben hin gedeckelt und eventuelle Überschüsse werden
anteilig als Rücklage oder als Bonuszahlung für die letzten Nutzer im Rentensystem
zurückgelegt. 

\begin{equation}
	total\_units\_month = 
	\sum_{n=1}^{count} units
\end{equation}

\begin{equation}
	total\_weighted\_dpt = 
	\sum_{n=1}^{count} pension\_user\_dpt
\end{equation}


\begin{equation}
CPR = \begin{cases} 
	avg(units) & _{(CPR > avg(units))} \\
	\frac{total\_units\_month  \cdot avg(units)} 
		{total\_weighted\_dpt} & _{(other)}
\end{cases}
\end{equation}



%\begin{equation}
%	pension = 	
%	weighted\_dpt\_units\_rate \cdot user\_dpt
%\end{equation}


\subsubsection*{Rücklagenrentenrate ($SPR$)}

Wenn $weighted\_dpt\_units\_rate < avg(units)$ dann 

\begin{equation}
	active\_user\_dpts = 
	\sum_{n=1}^{count} active\_user\_dpt
\end{equation}

\begin{equation}
	total\_units = 	
	\sum_{p=1}^{periods} \sum_{n=1}^{users} user\_units	
\end{equation}

\begin{equation}
	savings\_dpt\_units\_rate = 	
	\frac{total\_units} 
	{active\_user\_dpts \cdot years \cdot 12}		
\end{equation}

\begin{equation}
	pension = savings\_dpt\_units\_rate \cdot user\_dpt
\end{equation}


\subsubsection*{Bonusrentenrate ($LPR$)}

Wenn $entitlementMonths >= 1$ dann 

\begin{equation*}
	active\_user\_dpts = 
	\sum_{n=1}^{count} active\_user\_dpt
\end{equation*}

\begin{equation*}
	total\_units = 	
	\sum_{p=1}^{periods} \sum_{n=1}^{users} user\_units	
\end{equation*}

\begin{equation*}
	laggards\_rate = 	
	\frac{total\_units} 
	{active\_user\_dpts \cdot years \cdot 12}		
\end{equation*}

\begin{equation*}
	pension = laggards\_rate \cdot user\_dpt
\end{equation*}


Voraussetzung ist gegeben wenn $pensioners / periods_open$ genau 1 entspricht